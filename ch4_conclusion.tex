% まずはじめに \documentclass を指定する
\documentclass[uplatex]{jsbook}
%
% Packages
%
\usepackage[version=3]{mhchem}
\usepackage{geometry} % 余白の調整用
\usepackage{amsmath,amssymb}
\usepackage{booktabs} % 表組みのパッケージ
\usepackage{bm}
\usepackage[dvipdfmx]{graphicx}
\usepackage[subrefformat=parens]{subcaption}
\usepackage{ascmac}
\usepackage{braket} % ブラケット記法
\usepackage{cite} % [n-m]形式の引用を用いるため
%\usepackage{fancyhdr} % ヘッダ
\usepackage{feynmp} % Feynmanダイアグラムを書くため
\usepackage{color} % \debug 用に文字色をつける
\usepackage{setspace} % setstretchを使うため
\usepackage{newtxtext,newtxmath} % Times系フォントの使用

%\usepackage{abstract}
% 以下4つのパッケージはbibの著者名に出てきた欧文文字の文字化け対策として入れたが、
% 吉田は↓で上手くいく理由を良く理解していない。
% よく分からないパッケージは、それ無しで動くならコメントアウトしておくこと推奨。
\usepackage[uplatex,deluxe]{otf} %  \usepackage[prefernoncjk]{pxcjkcat} より先に読み込むべし(according to pxcjkcatのサイト)
\usepackage[prefernoncjk]{pxcjkcat}
\usepackage[T1]{fontenc} % T1エンコーディング, Bibliographyの著者用
%\usepackage[utf8]{inputenc}
%
% .bst ファイルは end.tex 内で指定することにしたから、次の行はコメントアウト
%\bibliographystyle{osajnlt}
%
% 行間と余白の調整
\setstretch{1.25}
\geometry{top=3truecm,bottom=3truecm,right=3truecm,left=3truecm}
%
% 見出しをセリフ・太字にする
\renewcommand{\headfont}{\bfseries}
%
% 脚注番号を記号に変える
\renewcommand{\thefootnote}{\fnsymbol{footnote}}
% 脚注記号を改ページでリセットする
\makeatletter
\@addtoreset{footnote}{page}
\makeatother
%
%「参考文献」を「References」にする
\renewcommand{\bibname}{References}
%「目次」を「Contents」にする
\renewcommand{\contentsname}{Contents}
%「第\CID{1624}章」を「Chapter x」にする
\renewcommand{\prechaptername}{Chapter }
\renewcommand{\postchaptername}{}

%表のキャプションを 表1.1ではなく Tabel1.1にする
\renewcommand{\tablename}{Table}
\renewcommand{\figurename}{Figure}


%% Appendix re-define
\renewcommand{\appendixname}{Appendix~}

%%目次にsabsectionを表示する
\setcounter{tocdepth}{2}

%
% 
\newcommand{\refeq}[1]{Eq.\,(\ref{#1})}
\renewcommand\vec\bm
\newcommand{\RR}{\vec{r}}
\newcommand{\PP}{\vec{p}}
\newcommand{\QQ}{\vec{q}}
\newcommand{\KK}{\vec{k}}
\newcommand{\TT}{\vec{t}}
\renewcommand{\SS}{\vec{s}}
\newcommand{\fermion}{\hat{\psi}}
\newcommand{\boson}{\hat{\phi}}
\newcommand{\BB}{\hat{b}}
\newcommand{\CC}{\hat{c}}
\newcommand{\momint}[1]{\frac{d^3{#1}}{(2\pi)^3}}
\newcommand{\sh}{\mathrm{Y}}
\newcommand{\schrodinger}{Schr\"{o}dinger }
\newcommand{\Tr}{\mathrm{Tr}}
\newcommand{\tr}{\mathrm{tr}}
\newcommand{\chrom}{\ce{Cr2O3}}
%
\newcommand{\debug}[1]{\textcolor{red}
{\textbf{[#1]}}}
\newcommand{\atom}[2][]{{}^{#1}\mathrm{#2}}
%
%
%

\begin{document}
% 数式の上下の余白を詰める
\setlength\abovedisplayskip{6.5pt}
\setlength\belowdisplayskip{6.5pt}
%
% Contents
%
\chapter{Conclusion}
\label{conclusion}

\section{Summary}
In this thesis, we used the extended SCDFT scheme including the effect of spin fluctuations
and explored this effect on superconducting transition temperatures ($T_c$)
of V, Nb and Al from first-principles. 
As a result, it is confirmed that the effect of spin fluctuations commonly
reduces the $T_c$ of transition metals. 
For V having $3d$ electrons, $T_c$ with the effect of spin fluctuations is 8.3K, while it 
without this effect is 17.8K.
For Nb having $4d$ electrons, $T_c$ with the effect of spin fluctuations is 8.9K, while it 
without this effect is 11.1K.
The agreement of calculated $T_c$s between the experimental values is improved for both of these 
two materials. On the other hand, the $T_c$ reduction is not observed for Al.
According to these results, it can be deduced that the reduction of $T_c$ due to spin fluctuations
is related to the magnitude of the electronic localization.

\section{Future issues}

\begin{description}
	\item[Formalism] \mbox{}\\
We found that the recently developed formalism has a problem with respect to the frequency dependence
of the effective interaction mediated by spin fluctuations. Therefore, we {\it ad hoc} neglected
one problematic term $f^{\rm xc}$. 
There are some possible causes of the problematic term. One direct cause is the adiabatic 
approximation for $f^{\rm xc}$. This approximation makes $f^{\rm xc}$ frequency-independent(Chapter 
\ref{application}).
Other possible causes are the choice of the initial self-energy (see (\ref{eq:localapp})) and 
the fact that we used the particle-hole propagator in order to take into account the effect of
spin fluctuations. In practice, these assumptions are inevitable to avoid the numerical difficulty 
to handle four-point functions. Some approximations which can handle the effect of spin 
fluctuations with two-point functions without involving the problem of frequency dependence may 
overcome this difficulty.

\item[Application] \mbox{}\\
	In this thesis it is showed that $T_c$ of some materials are estimated more accurately by 
	including the effect of spin fluctuations from first-principles. 
	In principle, this extension can be applied to any 
	kind of superconductors because the spin fluctuations are quite general phenomena.
	At first, this extended scheme is developed in order to study the pairing mechanism in 
	Fe-based superconductors and already applied to FeSe\cite{Essenberger2016, Lischner2015}. 
	Not only in case of Fe-based superconductors, this extended scheme is helpful to understand 
	the mechanism of certain high-$T_c$ superconductors.


\end{description}

\bibliographystyle{osajnl}
\bibliography{library_intro,library_method,library_result}

\end{document}
