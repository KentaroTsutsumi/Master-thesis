% まずはじめに \documentclass を指定する
\documentclass[uplatex]{jsbook}
%
% Packages
%
\usepackage[version=3]{mhchem}
\usepackage{geometry} % 余白の調整用
\usepackage{amsmath,amssymb}
\usepackage{booktabs} % 表組みのパッケージ
\usepackage{bm}
\usepackage[dvipdfmx]{graphicx}
\usepackage[subrefformat=parens]{subcaption}
\usepackage{ascmac}
\usepackage{braket} % ブラケット記法
\usepackage{cite} % [n-m]形式の引用を用いるため
%\usepackage{fancyhdr} % ヘッダ
\usepackage{feynmp} % Feynmanダイアグラムを書くため
\usepackage{color} % \debug 用に文字色をつける
\usepackage{setspace} % setstretchを使うため
\usepackage{newtxtext,newtxmath} % Times系フォントの使用

%\usepackage{abstract}
% 以下4つのパッケージはbibの著者名に出てきた欧文文字の文字化け対策として入れたが、
% 吉田は↓で上手くいく理由を良く理解していない。
% よく分からないパッケージは、それ無しで動くならコメントアウトしておくこと推奨。
\usepackage[uplatex,deluxe]{otf} %  \usepackage[prefernoncjk]{pxcjkcat} より先に読み込むべし(according to pxcjkcatのサイト)
\usepackage[prefernoncjk]{pxcjkcat}
\usepackage[T1]{fontenc} % T1エンコーディング, Bibliographyの著者用
%\usepackage[utf8]{inputenc}
%
% .bst ファイルは end.tex 内で指定することにしたから、次の行はコメントアウト
%\bibliographystyle{osajnlt}
%
% 行間と余白の調整
\setstretch{1.25}
\geometry{top=3truecm,bottom=3truecm,right=3truecm,left=3truecm}
%
% 見出しをセリフ・太字にする
\renewcommand{\headfont}{\bfseries}
%
% 脚注番号を記号に変える
\renewcommand{\thefootnote}{\fnsymbol{footnote}}
% 脚注記号を改ページでリセットする
\makeatletter
\@addtoreset{footnote}{page}
\makeatother
%
%「参考文献」を「References」にする
\renewcommand{\bibname}{References}
%「目次」を「Contents」にする
\renewcommand{\contentsname}{Contents}
%「第\CID{1624}章」を「Chapter x」にする
\renewcommand{\prechaptername}{Chapter }
\renewcommand{\postchaptername}{}

%表のキャプションを 表1.1ではなく Tabel1.1にする
\renewcommand{\tablename}{Table}
\renewcommand{\figurename}{Figure}


%% Appendix re-define
\renewcommand{\appendixname}{Appendix~}

%%目次にsabsectionを表示する
\setcounter{tocdepth}{2}

%
% 
\newcommand{\refeq}[1]{Eq.\,(\ref{#1})}
\renewcommand\vec\bm
\newcommand{\RR}{\vec{r}}
\newcommand{\PP}{\vec{p}}
\newcommand{\QQ}{\vec{q}}
\newcommand{\KK}{\vec{k}}
\newcommand{\TT}{\vec{t}}
\renewcommand{\SS}{\vec{s}}
\newcommand{\fermion}{\hat{\psi}}
\newcommand{\boson}{\hat{\phi}}
\newcommand{\BB}{\hat{b}}
\newcommand{\CC}{\hat{c}}
\newcommand{\momint}[1]{\frac{d^3{#1}}{(2\pi)^3}}
\newcommand{\sh}{\mathrm{Y}}
\newcommand{\schrodinger}{Schr\"{o}dinger }
\newcommand{\Tr}{\mathrm{Tr}}
\newcommand{\tr}{\mathrm{tr}}
\newcommand{\chrom}{\ce{Cr2O3}}
%
\newcommand{\debug}[1]{\textcolor{red}
{\textbf{[#1]}}}
\newcommand{\atom}[2][]{{}^{#1}\mathrm{#2}}
%
%
%

\begin{document}
% 数式の上下の余白を詰める
\setlength\abovedisplayskip{6.5pt}
\setlength\belowdisplayskip{6.5pt}
%
% Contents
%
\chapter{Application to elemental metals}
\label{application}
In this section, we show our calculation results including the effect of spin fluctuations 
for V, Nb and Al. We show that, in the case of V, the magnitude of the kernel originating 
from spin fluctuations $\mathcal K^{\rm SF}$ is comparable to that of Coulomb interaction.
Accordingly, the estimated $T_{c}$ is significantly lower than that without including the
effect of spin fluctuations. We also show that the magnitude of $\mathcal K^{\rm SF}$ and the 
resulting amount of $T_{c}$ reduction is related to the electronic localization.

\section{Computational details}
We developed 
We used the Quantum ESPRESSO package\cite{QE} to obtain the Kohn-Sham energies and wave functions.
We obtained the phonon frequencies and electron-phonon coupling by applying the density functional
perturbation theory(DFPT)\cite{DFPT}.
We used the GGA-PBE exchange-correlation functional\cite{GGAPBE} and the norm-conserving 
pseudopotentials\cite{normcons} for all materials. 
The scalar relativistic correction was applied to the Nb pseudopotential.
The numerical conditions are listed in Table\ref{tab:numcond}.
We used the first order Hermite-Gaussian smearing function\cite{Paxton1989} 
with width = $0.025$Ry for metallic calculations.

\begin{table}[hbtp]
	\centering
	\caption{Numerical conditions.}
	\begin{tabular}{lccc}
		\hline \hline
		& V & Nb & Al \\
		\hline
	$k$ grid (structure optimization) & $12\times12\times12$ & $12\times12\times12$ & $12\times12\times12$ \\
	$q$ grid (dynamical matrices) & $ 8\times8\times8$ & $8\times8\times8$ & $8\times8\times8$ \\
	$k$ grid (phonon linewidth) & $48\times48\times48$ & $48\times48\times48$ & $48\times48\times48$ \\
	wavefunction energy cutoff & 80Ry & 80Ry & 20Ry \\
	charge density energy cutoff & 320Ry & 320Ry & 80Ry \\
	\hline \hline
	\end{tabular}
	\label{tab:numcond}
\end{table}

\section{Results and discussion}
Calculated results are summarized in table\ref{tab:results}.
The lattice constants resulting from the structure optimization agree with the experimental ones.
We used the theoretically optimized lattice constants in the following calculations.
The results labeled static are obtained by neglecting the frequency dependence of the 
screened Coulomb interaction and spin fuctuations-mediated interaction.
We applied the ALDA for the screened Coulomb interaction as mentioned in the previous chapter, 
while RPA is applied in Ref.\cite{RA2013}.

In case of V, we obtain $T_c = 17.8$K with dinamical calculation when we do not take account of 
spin fluctuations and this value is extremely higher than the experimental value.
On the other hand, when we include the spin fluctuations, $T_c$ is reduced to $8.3$K.
Although this value is still higher than exprerimental one by about 60\%, the agreement with 
the experimental value is improved.
In case of Nb, we obtain $T_c = 11.1$K with dinamical calculation when we do not take account of 
spin fluctuations and this value is higher than experimental value by about 15\%.
If the effect of spin fluctuations is included, $T_c$ is reduced to $8.9$K.
The qualitative trend that the $T_c$ is reduced due to the effect of spin fluctuations is the same as 
in the case of V, but the amount of reduction of $T_c$ is relatively smaller than that of V.
This quantitatively different behavior comes from the strength of electronic localization in each material.

The ferromagnetic fluctuations come from the exchange effect\cite{Berk1966}: two electrons having the 
same spins are 
required to be separated in space due to the Pauli principle. As a result, in a 
nearly homogeneous system, these two electron have lower energy than that of two electrons having 
opposite spins and the ferromagnetic fluctuations are induced.
In addition, the exchange effect is significant in strongly localized electrons.
Returning to the present results, there are 4$d$ electrons in Nb, while there are 3$d$ electrons 
in V and they are more locarized than 4$d$ electrons. Therefore, the effect of spin fluctuations
is stronger in V than in Nb.
The result in Al can be interpreted similarly. In case of Al, the $T_c$ reduction due to spin fluctuations
is not observed. This result is reasonable because there are nearly free electrons in Al and the
exchange effect is relatively smaller than in the case of V and Nb.
%
\begin{table}[htbp]
	\centering
	\caption{Calculation results. SF means spin fluctuations. Values of $\mu^{\ast}$ are 
	calculated with McMillan-Allen-Dynes formula using $\lambda$, $\omega_{\rm ln}$ and 
	$T_c$ calculated with SCDFT.} 
	\begin{tabular}{lccc}
		\hline \hline
		& V & Nb & Al \\
		\hline
		lattice constant $a$ [\AA] & 3.00 & 3.28 & 4.04 \\
		lattice constant $a$ [\AA] (experiment) & 3.02\cite{Kuentzler1985} & 3.31\cite{Laesser1985} & 4.04\cite{Sumiyama1990} \\
		\hline
		$\lambda$ & 1.31 & 1.39 & 0.42 \\
		$\omega_{\rm ln}$ [K] & 212 & 175 & 298 \\
		\hline
		$T_c$ (static, without SF) [K] & 10.5 & 7.4 & 0.5 \\
		$T_c$ (dynamical, without SF) [K] & 17.8 & 11.1 & 1.4 \\
		$T_c$ (static, with SF) [K] & 5.0 & 6.2 & 0.4 \\
		$T_c$ (dynamical, with SF) [K] & 8.3 & 8.9 & 1.4 \\
		$T_c$ (Thomas Fermi, without SF) [K] & - & 9.5\cite{Luders2005} & 0.14\cite{Luders2005} \\
		$T_c$ (static RPA, without SF) [K] & - & - & 0.8\cite{RA2013} \\
		$T_c$ (dynamical RPA, without SF) [K] & - & - & 1.4\cite{RA2013} \\
		$T_c$ (experimental)\cite{Ashcroft} [K] & 5.38 & 9.50 & 1.14 \\
		\hline
		$\mu^{\ast}$ (SCDFT, static,without SF) & 0.252 & 0.299 & 0.142 \\
		$\mu^{\ast}$ (SCDFT, dynamical, without SF) & 0.144 & 0.228 & 0.105 \\
		$\mu^{\ast}$ (SCDFT, static, with SF) & 0.349 & 0.324 & 0.149 \\
		$\mu^{\ast}$ (SCDFT, dynamical, with SF) & 0.288 & 0.269 & 0.105 \\
		\hline \hline
	\end{tabular}
	\label{tab:results}
\end{table}
%

We found that the effect of spin fluctuations on $T_c$ is at least qualitatively reasonable and 
the amount of the $T_c$ reduction is related to the electronic localization.
In order to investigate how spin fluctuations affect the $T_c$ closely, we 
consider the partially averaged nondiagonal exchange correlation kernel which is defined as
%
\begin{equation}
	\mathcal K_{n\bm k}(\xi) \equiv \frac{1}{N(\xi)}
	\sum_{n'\bm k'} \mathcal K_{n\bm k n'\bm k'}\delta(\xi - \xi_{n'\bm k'}).
	\label{eq:aveK}
\end{equation}
%

The averaged exchange correlation kernels and corresponding gap functions calculated for V 
are plotted in Fig.\ref{fig:Vaveker}. The kernel originating from the electron-phonon interaction
$\mathcal K^{\rm ph}$ is negative within the low energy region and nearly zero in the region 
where $\xi \gtrsim 1$eV. The kernels of the static Coulomb interaction is nearly constant in the
whole energy region. On the other hand, the kernel stemmed from the plasmon-induced dynamical
Coulomb interaction rises from nearly zero to positive around $\xi \approx 10^{-1}$eV.
This contribution suppresses the effective repulsion between quasiparticles due to the 
retardation effect and the resulting $T_c$ is enhanced compared to that of static approximation
\cite{RA2013}.
The kernel originating from dynamical spin fluctuations is nearly positive constant in the energy region 
In the high-energy region, on the other hand, the contribution from spin fluctuations reduces to zero. 
Furthermore, the magnitude of repulsion due to spin fluctuations in low-energy region is comparative 
to that of the static Coulomb interaction. This energy dependence of
$\mathcal K^{\rm SF}$ suppresses the low-energy positive gap function (Figs.\ref{fig:Vaveker}(b)
and \ref{fig:VgapT}) and
the resulting $T_c$ strongly.
It should be noted that a cusp is observed in the gap function with spin fluctuations 
around $\xi \approx 10^{-2}{\rm eV}$. This cusp may be caused by the result that there is a small
peak in spin fluctuations kernel between $\xi = 10^{-1}{\rm eV}$ and $10^{-2}{\rm eV}$ and
$\mathcal K^{\rm SF}$ is slightly reduced in low-energy region.

%
\begin{figure}[h]
	\centering
	\includegraphics[width=13truecm,clip]{../figure/result/V_kernels.pdf}
	\caption{(a) Decomposition of the averaged exchange-correlation kernels defined in 
		(\ref{eq:aveK}) at $T=0.01$K in V. (b) Corresponding gap funtions without 
	and with SF. $\xi$ is the energy measured from the chemical potential $\mu$.}
	\label{fig:Vaveker}
\end{figure}
%
\begin{figure}[h]
	\centering
	\includegraphics[width=10truecm,clip]{../figure/result/V_gap_mod.eps}
	\caption{Temperature dependence of the gap functions calculated with and without 
		the spin fluctuations kernel $\mathcal K^{\rm SF}$ and the experimental $T_c$ in V.
		An arrow indicates the experimental $T_c = 5.38{\rm K}$.}
	\label{fig:VgapT}
\end{figure}
%

Next, we see the averaged exchange-correlation kernels calculated for Nb in Fig.\ref{fig:Nbaveker}.
In the case of Nb, the qualitative trend is the same as in the case of V: The averaged kernel of spin fluctuations
is nearly positive constant in low energy region and reduces to zero in high energy region.
However, the magnitude of the kernel $\mathcal K^{\rm SF}$ is relatively smaller than that of V.
As noted above, this is due to the strength of the electronic localization.
%
\begin{figure}[h]
	\centering
	\includegraphics[width=13truecm,clip]{../figure/result/Nb_kernels.pdf}
	\caption{(a) Decomposition of the averaged exchange-correlation kernels
		at $T=0.01$K in Nb. (b) Corresponding gap funtions without 
	and with SF.}
	\label{fig:Nbaveker}
\end{figure}
%
\begin{figure}[h]
	\centering
	\includegraphics[width=10truecm,clip]{../figure/result/Nb_gap_mod.eps}
	\caption{Temperature dependence of the gap functions calculated with and without 
		the spin fluctuations kernel $\mathcal K^{\rm SF}$ and the experimental $T_c$ in Nb.
		An arrow indicates the experimental $T_c = 9.5{\rm K}$.}
	\label{fig:NbgapT}
\end{figure}

Finally, we see the averaged exchange-correlation kernels calculated for Al in Fig.\ref{fig:Alaveker}.
The kernel $\mathcal K^{\rm SF}$ is significantly smaller than that of the static Coulomb interaction.
Therefore, the corresponding gap function and the resulting $T_c$ is not so modified.
%
% kokokara DISCUSSION deha ?????
%
\begin{figure}[h]
	\centering
	\includegraphics[width=13truecm,clip]{../figure/result/Al_kernels.pdf}
	\caption{(a) Decomposition of the averaged exchange-correlation kernels
		at $T=0.01$K in Al. (b) Corresponding gap funtions without 
	and with SF.}
	\label{fig:Alaveker}
\end{figure}
%

Here we should note the expression of the spin fluctuations-mediated interaction (\ref{eq:finallambda}).
As mentioned in the previous chapter, we neglected the linear term of $f^{\rm xc}$ defined as 
(\ref{eq:deffxc}). If we include the linear term, we observe an unreasonable result in the
calculation for Al: The gap function is enhanced in low energy region and the resulting $T_c$
is also enhanced. It can be considered that this strange result may comes from the frequency
dependence of $\Lambda^{\rm SF}$. Here we show the expression of $\Lambda^{\rm SF}$ in crystals with 
the linear term which is written with the double-Fourier transform\cite{Hybertsen1987}
%
\begin{equation}
	\Lambda^{\rm SF}_{\mathbf G \mathbf G'}(\mathbf q, \omega) = f^{\rm xc}(\mathbf G' - \mathbf G) - 
	3\sum_{\mathbf G_1 \mathbf G_2} f^{\rm xc}(\mathbf G_1 - \mathbf G)\chi_{\mathbf G_1 \mathbf G_2}(\mathbf q, \omega)
	f^{\rm xc}(\mathbf G' - \mathbf G_2).
	\label{eq:lambdamiss}
\end{equation}
%
In the above expression, there is a frequency-dependent term and a frequency-independent term.
The reason why $f^{\rm xc}$ is frequency-independent is that we carried out 
%a local approximation
%to the starting self-energy $\Sigma^{V}$ (see (\ref{eq:localapp})). Within this approximation,
%since one is obliged to apply 
the adiabatic approximation to $f^{\rm xc}(\bm r_1,t_1, \bm r_2,t_2)$:
%
\begin{equation}
	f^{\rm xc}_{\rm adia}(\bm r_1,t_1, \bm r_2, t_2) = f^{\rm xc}(\bm r_1,\bm r_2, t_1)\delta(t_1 - t_2),
	\label{eq:adiabatic}
\end{equation}
%
$f^{\rm xc}$ becomes $\omega$-independent after the Fourier transform.

In the original paper by Essenberger\cite{Essenberger2014}, this term is neglected based on the following assumption.
It is assumed that this formalism is applied to the unconventional superconductors like Fe-based ones
in which magnetic fluctuations are nearly strong enough to reach the magnetic ordered phase.
In such cases, the second term in (\ref{eq:lambdamiss}) proportional to the spin 
susceptibility $\chi$ is dominant and the linear $f^{\rm xc}$ term can be neglected.

On the other hand, this assumption is not valid in the case of elemental metals such as Al and the 
linear $f^{\rm xc}$ term can not be neglected.
However, this $\omega$-independent term is physically unreasonable. 
We discuss the reason in anologous to the screened Coulomb interaction below.
In case of the frequency-dependent screened interaction, in the high-frequency limit, 
the screened Coulomb interaction reduces to the bare Coulomb interaction because
the polarization function reduces to zero (see (\ref{eq:chi0})).
This frequency-dependence reflects the fact that electrons behave as bare ones, namely, 
they do not feel any exchange-correlation effect in the high-frequency region.
According to this consideration, the effective interaction mediated by spin fluctuations originating
from the exchange-correlation effect should vanish in the high-frequency limit.
Therefore, the frequency-independent non-zero term $f^{\rm xc}$ is physically unreasonable and 
neglected in our calculations.
In fact, there are some theoretical studies on the frequency dependence of the functional derivative
of the exchange-correlation potential with respect to the electronic density\cite{Qian2002}.
On the other hand, similar analyses on the functional derivative with respect to the spin density is lacking.
If the function form with respect to frequency is developed, it may be able to take into account the 
linear $f^{\rm xc}$ term.
%
% kokomade DISCUSSION deha ?????
%
%
%%%%%%%%%%%%%%%%%%%%%  CRYSTAL STRUCTURE  %%%%%%%%%%%%%%%%
%\section{Table} \label{sec:Constructing the Model}
%
%%%%%%%%%%%%%%%%%%%%%%   FORCE CONSTANT  %%%%%%%%%%%%%%%%
%\subsection{Table}  \label{sec:Force Constant}
%Chapter reference  Fig\ref{fig:structure_copt}. 
%
%\begin{table}[hbtp] %%%%% TABLE FREAUENCIES
%  \centering
%  \begin{tabular}{lcccccccc}
%    \hline \hline
%    &\multicolumn{2}{c}{A$_{2u}$ modes} & & \multicolumn{4}{c}{E$_u$ modes } \\ 
%   \hline 
%    LDA+U (this work)  & 407  & 574  & & 311 & 447 & 562 & 635\\
%    PBE (Ref.\cite{Ye2014}) & 388 & 522 & & 297 & 427 & 510 & 610  \\
%    Expt. (Ref.\cite{Lucovsky1977})  & 402 & 533  & & 305 & 440 & 538 & 609 \\
%    \hline \hline
%  \end{tabular}
%    \caption{Phonon frequencies ($\rm{cm^{-1}}$) }
%\label{tab:freq}
%\end{table}

\bibliographystyle{osajnl}
\bibliography{library_intro,library_method,library_result}

\end{document}
