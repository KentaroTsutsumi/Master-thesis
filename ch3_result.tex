% まずはじめに \documentclass を指定する
\documentclass[uplatex]{jsbook}
%
% Packages
%
\usepackage[version=3]{mhchem}
\usepackage{geometry} % 余白の調整用
\usepackage{amsmath,amssymb}
\usepackage{booktabs} % 表組みのパッケージ
\usepackage{bm}
\usepackage[dvipdfmx]{graphicx}
\usepackage[subrefformat=parens]{subcaption}
\usepackage{ascmac}
\usepackage{braket} % ブラケット記法
\usepackage{cite} % [n-m]形式の引用を用いるため
%\usepackage{fancyhdr} % ヘッダ
\usepackage{feynmp} % Feynmanダイアグラムを書くため
\usepackage{color} % \debug 用に文字色をつける
\usepackage{setspace} % setstretchを使うため
\usepackage{newtxtext,newtxmath} % Times系フォントの使用

%\usepackage{abstract}
% 以下4つのパッケージはbibの著者名に出てきた欧文文字の文字化け対策として入れたが、
% 吉田は↓で上手くいく理由を良く理解していない。
% よく分からないパッケージは、それ無しで動くならコメントアウトしておくこと推奨。
\usepackage[uplatex,deluxe]{otf} %  \usepackage[prefernoncjk]{pxcjkcat} より先に読み込むべし(according to pxcjkcatのサイト)
\usepackage[prefernoncjk]{pxcjkcat}
\usepackage[T1]{fontenc} % T1エンコーディング, Bibliographyの著者用
%\usepackage[utf8]{inputenc}
%
% .bst ファイルは end.tex 内で指定することにしたから、次の行はコメントアウト
%\bibliographystyle{osajnlt}
%
% 行間と余白の調整
\setstretch{1.25}
\geometry{top=3truecm,bottom=3truecm,right=3truecm,left=3truecm}
%
% 見出しをセリフ・太字にする
\renewcommand{\headfont}{\bfseries}
%
% 脚注番号を記号に変える
\renewcommand{\thefootnote}{\fnsymbol{footnote}}
% 脚注記号を改ページでリセットする
\makeatletter
\@addtoreset{footnote}{page}
\makeatother
%
%「参考文献」を「References」にする
\renewcommand{\bibname}{References}
%「目次」を「Contents」にする
\renewcommand{\contentsname}{Contents}
%「第\CID{1624}章」を「Chapter x」にする
\renewcommand{\prechaptername}{Chapter }
\renewcommand{\postchaptername}{}

%表のキャプションを 表1.1ではなく Tabel1.1にする
\renewcommand{\tablename}{Table}
\renewcommand{\figurename}{Figure}


%% Appendix re-define
\renewcommand{\appendixname}{Appendix~}

%%目次にsabsectionを表示する
\setcounter{tocdepth}{2}

%
% 
\newcommand{\refeq}[1]{Eq.\,(\ref{#1})}
\renewcommand\vec\bm
\newcommand{\RR}{\vec{r}}
\newcommand{\PP}{\vec{p}}
\newcommand{\QQ}{\vec{q}}
\newcommand{\KK}{\vec{k}}
\newcommand{\TT}{\vec{t}}
\renewcommand{\SS}{\vec{s}}
\newcommand{\fermion}{\hat{\psi}}
\newcommand{\boson}{\hat{\phi}}
\newcommand{\BB}{\hat{b}}
\newcommand{\CC}{\hat{c}}
\newcommand{\momint}[1]{\frac{d^3{#1}}{(2\pi)^3}}
\newcommand{\sh}{\mathrm{Y}}
\newcommand{\schrodinger}{Schr\"{o}dinger }
\newcommand{\Tr}{\mathrm{Tr}}
\newcommand{\tr}{\mathrm{tr}}
\newcommand{\chrom}{\ce{Cr2O3}}
%
\newcommand{\debug}[1]{\textcolor{red}
{\textbf{[#1]}}}
\newcommand{\atom}[2][]{{}^{#1}\mathrm{#2}}
%
%
%

\begin{document}
% 数式の上下の余白を詰める
\setlength\abovedisplayskip{6.5pt}
\setlength\belowdisplayskip{6.5pt}
%
% Contents
%
\chapter{Application to elemental metals}
In this section, we show our results of the SCDFT including the effect of spin fluctuations 
in case of V, Nb and Al. We show that, in case of V, the magnitude of the kernel originated 
from spin fluctuations $\mathcal K^{\rm SF}$ is comparable to that of Coulomb interaction.
According to that, the estimated $T_{c}$ is significantly lower than that without including the
effect of spin fluctuations. We also show that the magnitude of $\mathcal K^{\rm SF}$ and the 
resulting amount of $T_{c}$ reduction is related to the electronic locarization.

\section{Computational details}
We used the Quantum ESPRESSO\cite{QE} package to obtain the Kohn-Sham energies and wave functions.
We obtained the phonon frequencies and electron-phonon coupling by using the density functional
perturbation theory(DFPT)\cite{DFPT}.
We used the GGA-PBE exchange-correlation functional\cite{GGAPBE} and the norm-conserving 
pseudopotentials\cite{normcons} for all materials. 
The scalar relativistic correction was applied to the Nb pseudopotential.
The numerical conditions are listed in Table\ref{tab:numcond}.
We used the first order of Hermite-Gaussian smearing function\cite{Paxton1989} 
with width = $0.025$Ry for metallic calculations.

\begin{table}[hbtp]
	\centering
	\caption{Numerical conditions.}
	\begin{tabular}{lccc}
		\hline \hline
		& V & Nb & Al \\
		\hline
	$k$ grid (structure optimization) & $12\times12\times12$ & $12\times12\times12$ & $12\times12\times12$ \\
	$q$ grid (dynamical matrices) & $ 8\times8\times8$ & $8\times8\times8$ & $8\times8\times8$ \\
	$k$ grid (phonon linewidth) & $48\times48\times48$ & $48\times48\times48$ & $48\times48\times48$ \\
	wavefunction energy cutoff & 80Ry & 80Ry & 20Ry \\
	charge density energy cutoff & 320Ry & 320Ry & 80Ry \\
	\hline \hline
	\end{tabular}
	\label{tab:numcond}
\end{table}

\section{Results}
Calculated results are summarized in table\ref{tab:results}.
The lattice constants resulting from the structure optimization agree with the experimental ones.
We used the theoretically optimized lattice constants in the following calculations.
The results labeled static are obtained by neglecting the frequency dependence of the 
screened Coulomb interaction and spin fuctuations-mediated interaction.
We applied the ALDA for the screened Coulomb interaction as mentioned in the previous chapter, 
while RPA is applied in Ref.\cite{RA2013}.

In case of V, we obtain $T_c = 17.8$K with dinamical calculation when we did not take account of 
spin fluctuations and this value is extremely higher than experimental value.
On the other hand, when we include the spin fluctuations, $T_c$ is reduced to $8.3$K.
Although this value is still higher than exprerimental one by about 60\%, the agreement between 
the experimental value is improved.
In case of Nb, we obtain $T_c = 11.1$K with dinamical calculation when we did not take account of 
spin fluctuations and this value is higher than experimental value by about 15\%.
On the other hand, when we include the spin fluctuations, $T_c$ is reduced to $8.9$K.
The qualitative trend that the $T_c$ is reduced due to the effect of spin fluctuations is same as 
in case of V, but the amount of reduction of $T_c$ is relatively smaller than in case of V.
This quantitatively different behavior comes from the strength of electronic locarization in each material.

The ferromagnetic fluctuations come from the exchange effect: Two electrons having same spins are 
required to be saparated in space due to the Pauli's rule. As a result, in the 
nearly homogeneous system, these two electron have lower energy than that of two electrons having 
opposite spins and the ferromagnetic fluctuations are induced.
In addition, the exchange effect is significant in strongly localized electrons.
Returning to the present results, there are 4$d$ electrons in Nb, while there are 3$d$ electrons 
in V and they are strongly locarized than 4$d$ electrons. 

%
\begin{table}[hbtp]
	\centering
	\caption{Calculation results. SF means spin fluctuations.} 
	\begin{tabular}{lccc}
		\hline \hline
		& V & Nb & Al \\
		\hline
		lattice constant $a$ [\AA] & 3.00 & 3.28 & 4.04 \\
		lattice constant $a$ [\AA] (experiment) & 3.02\cite{Kuentzler1985} & 3.31\cite{Laesser1985} & 4.04\cite{Sumiyama1990} \\
		\hline
		$\lambda$ & 1.31 & 1.39 & 0.42 \\
		$\omega_{\rm ln}$ & 212 & 175 & 298 \\
		\hline
		$T_c$ (static, without SF) [K] & 10.5 & 7.4 & 0.5 \\
		$T_c$ (dynamical, without SF) [K] & 17.8 & 11.1 & 1.4 \\
		$T_c$ (static, with SF) [K] & 5.0 & 6.2 & 0.4 \\
		$T_c$ (dynamical, with SF) [K] & 8.3 & 8.9 & 1.4 \\
		$T_c$ (Thomas Fermi, without SF) [K] & - & 9.5\cite{Luders2005} & 0.14\cite{Luders2005} \\
		$T_c$ (static RPA, without SF) [K] & - & - & 0.8\cite{RA2013} \\
		$T_c$ (dynamical RPA, without SF) [K] & - & - & 1.4\cite{RA2013} \\
		$T_c$ (experimental)\cite{Ashcroft} [K] & 5.38 & 9.50 & 1.14 \\
		\hline \hline
	\end{tabular}
	\label{tab:results}
\end{table}
%
%%%%%%%%%%%%%%%%%%%%%  CRYSTAL STRUCTURE  %%%%%%%%%%%%%%%%
%\section{Table} \label{sec:Constructing the Model}
%
%%%%%%%%%%%%%%%%%%%%%%   FORCE CONSTANT  %%%%%%%%%%%%%%%%
%\subsection{Table}  \label{sec:Force Constant}
%Chapter reference  Fig\ref{fig:structure_copt}. 
%
%\begin{table}[hbtp] %%%%% TABLE FREAUENCIES
%  \centering
%  \begin{tabular}{lcccccccc}
%    \hline \hline
%    &\multicolumn{2}{c}{A$_{2u}$ modes} & & \multicolumn{4}{c}{E$_u$ modes } \\ 
%   \hline 
%    LDA+U (this work)  & 407  & 574  & & 311 & 447 & 562 & 635\\
%    PBE (Ref.\cite{Ye2014}) & 388 & 522 & & 297 & 427 & 510 & 610  \\
%    Expt. (Ref.\cite{Lucovsky1977})  & 402 & 533  & & 305 & 440 & 538 & 609 \\
%    \hline \hline
%  \end{tabular}
%    \caption{Phonon frequencies ($\rm{cm^{-1}}$) }
%\label{tab:freq}
%\end{table}

\bibliographystyle{osajnl}
\bibliography{library_intro,library_method,library_result}

\end{document}
