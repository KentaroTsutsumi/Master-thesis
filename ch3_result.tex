% まずはじめに \documentclass を指定する
\documentclass[uplatex]{jsbook}
%
% Packages
%
\usepackage[version=3]{mhchem}
\usepackage{geometry} % 余白の調整用
\usepackage{amsmath,amssymb}
\usepackage{booktabs} % 表組みのパッケージ
\usepackage{bm}
\usepackage[dvipdfmx]{graphicx}
\usepackage[subrefformat=parens]{subcaption}
\usepackage{ascmac}
\usepackage{braket} % ブラケット記法
\usepackage{cite} % [n-m]形式の引用を用いるため
%\usepackage{fancyhdr} % ヘッダ
\usepackage{feynmp} % Feynmanダイアグラムを書くため
\usepackage{color} % \debug 用に文字色をつける
\usepackage{setspace} % setstretchを使うため
\usepackage{newtxtext,newtxmath} % Times系フォントの使用

%\usepackage{abstract}
% 以下4つのパッケージはbibの著者名に出てきた欧文文字の文字化け対策として入れたが、
% 吉田は↓で上手くいく理由を良く理解していない。
% よく分からないパッケージは、それ無しで動くならコメントアウトしておくこと推奨。
\usepackage[uplatex,deluxe]{otf} %  \usepackage[prefernoncjk]{pxcjkcat} より先に読み込むべし(according to pxcjkcatのサイト)
\usepackage[prefernoncjk]{pxcjkcat}
\usepackage[T1]{fontenc} % T1エンコーディング, Bibliographyの著者用
%\usepackage[utf8]{inputenc}
%
% .bst ファイルは end.tex 内で指定することにしたから、次の行はコメントアウト
%\bibliographystyle{osajnlt}
%
% 行間と余白の調整
\setstretch{1.25}
\geometry{top=3truecm,bottom=3truecm,right=3truecm,left=3truecm}
%
% 見出しをセリフ・太字にする
\renewcommand{\headfont}{\bfseries}
%
% 脚注番号を記号に変える
\renewcommand{\thefootnote}{\fnsymbol{footnote}}
% 脚注記号を改ページでリセットする
\makeatletter
\@addtoreset{footnote}{page}
\makeatother
%
%「参考文献」を「References」にする
\renewcommand{\bibname}{References}
%「目次」を「Contents」にする
\renewcommand{\contentsname}{Contents}
%「第\CID{1624}章」を「Chapter x」にする
\renewcommand{\prechaptername}{Chapter }
\renewcommand{\postchaptername}{}

%表のキャプションを 表1.1ではなく Tabel1.1にする
\renewcommand{\tablename}{Table}
\renewcommand{\figurename}{Figure}


%% Appendix re-define
\renewcommand{\appendixname}{Appendix~}

%%目次にsabsectionを表示する
\setcounter{tocdepth}{2}

%
% 
\newcommand{\refeq}[1]{Eq.\,(\ref{#1})}
\renewcommand\vec\bm
\newcommand{\RR}{\vec{r}}
\newcommand{\PP}{\vec{p}}
\newcommand{\QQ}{\vec{q}}
\newcommand{\KK}{\vec{k}}
\newcommand{\TT}{\vec{t}}
\renewcommand{\SS}{\vec{s}}
\newcommand{\fermion}{\hat{\psi}}
\newcommand{\boson}{\hat{\phi}}
\newcommand{\BB}{\hat{b}}
\newcommand{\CC}{\hat{c}}
\newcommand{\momint}[1]{\frac{d^3{#1}}{(2\pi)^3}}
\newcommand{\sh}{\mathrm{Y}}
\newcommand{\schrodinger}{Schr\"{o}dinger }
\newcommand{\Tr}{\mathrm{Tr}}
\newcommand{\tr}{\mathrm{tr}}
\newcommand{\chrom}{\ce{Cr2O3}}
%
\newcommand{\debug}[1]{\textcolor{red}
{\textbf{[#1]}}}
\newcommand{\atom}[2][]{{}^{#1}\mathrm{#2}}
%
%
%

\begin{document}
% 数式の上下の余白を詰める
\setlength\abovedisplayskip{6.5pt}
\setlength\belowdisplayskip{6.5pt}
%
% Contents
%
\chapter{Application to elemental metals}
In this section, we show our results of the SCDFT including the effect of spin fluctuations 
in case of V, Nb and Al. We show that, in case of V, the magnitude of the kernel originated 
from spin fluctuations $\mathcal K^{\rm SF}$ is comparable to that of Coulomb interaction.
According to that, the estimated $T_{c}$ is significantly lower than that without including the
effect of spin fluctuations. We also show that the magnitude of $\mathcal K^{\rm SF}$ and the 
resulting amount of $T_{c}$ reduction is related to the electronic locarization.

\section{Computational details}
We used Quantum ESPRESSO\cite{QE} to obtain the Kohn-Sham energies and wave functions.
We obtained the phonon frequencies and electron-phonon coupling by using the density functional
perturbation theory(DFPT)\cite{DFPT}.
We applied the optimized tetrahedron method\cite{opttetra} to execute the Brillouin zone
integration for the Kohn-Sham polarization function.

%%%%%%%%%%%%%%%%%%%%%  CRYSTAL STRUCTURE  %%%%%%%%%%%%%%%%
\section{Table} \label{sec:Constructing the Model}

%%%%%%%%%%%%%%%%%%%%%   FORCE CONSTANT  %%%%%%%%%%%%%%%%
\subsection{Table}  \label{sec:Force Constant}
Chapter reference  Fig\ref{fig:structure_copt}. 

\begin{table}[hbtp] %%%%% TABLE FREAUENCIES
  \centering
  \begin{tabular}{lcccccccc}
    \hline \hline
    &\multicolumn{2}{c}{A$_{2u}$ modes} & & \multicolumn{4}{c}{E$_u$ modes } \\ 
   \hline 
    LDA+U (this work)  & 407  & 574  & & 311 & 447 & 562 & 635\\
    PBE (Ref.\cite{Ye2014}) & 388 & 522 & & 297 & 427 & 510 & 610  \\
    Expt. (Ref.\cite{Lucovsky1977})  & 402 & 533  & & 305 & 440 & 538 & 609 \\
    \hline \hline
  \end{tabular}
    \caption{Phonon frequencies ($\rm{cm^{-1}}$) }
\label{tab:freq}
\end{table}

\bibliographystyle{osajnl}
\bibliography{library_intro,library_method,library_result}

\end{document}
