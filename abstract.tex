% まずはじめに \documentclass を指定する
\documentclass[uplatex]{jsbook}
%
% Packages
%
\usepackage[version=3]{mhchem}
\usepackage{geometry} % 余白の調整用
\usepackage{amsmath,amssymb}
\usepackage{booktabs} % 表組みのパッケージ
\usepackage{bm}
\usepackage[dvipdfmx]{graphicx}
\usepackage[subrefformat=parens]{subcaption}
\usepackage{ascmac}
\usepackage{braket} % ブラケット記法
\usepackage{cite} % [n-m]形式の引用を用いるため
%\usepackage{fancyhdr} % ヘッダ
\usepackage{feynmp} % Feynmanダイアグラムを書くため
\usepackage{color} % \debug 用に文字色をつける
\usepackage{setspace} % setstretchを使うため
\usepackage{newtxtext,newtxmath} % Times系フォントの使用

%\usepackage{abstract}
% 以下4つのパッケージはbibの著者名に出てきた欧文文字の文字化け対策として入れたが、
% 吉田は↓で上手くいく理由を良く理解していない。
% よく分からないパッケージは、それ無しで動くならコメントアウトしておくこと推奨。
\usepackage[uplatex,deluxe]{otf} %  \usepackage[prefernoncjk]{pxcjkcat} より先に読み込むべし(according to pxcjkcatのサイト)
\usepackage[prefernoncjk]{pxcjkcat}
\usepackage[T1]{fontenc} % T1エンコーディング, Bibliographyの著者用
%\usepackage[utf8]{inputenc}
%
% .bst ファイルは end.tex 内で指定することにしたから、次の行はコメントアウト
%\bibliographystyle{osajnlt}
%
% 行間と余白の調整
\setstretch{1.25}
\geometry{top=3truecm,bottom=3truecm,right=3truecm,left=3truecm}
%
% 見出しをセリフ・太字にする
\renewcommand{\headfont}{\bfseries}
%
% 脚注番号を記号に変える
\renewcommand{\thefootnote}{\fnsymbol{footnote}}
% 脚注記号を改ページでリセットする
\makeatletter
\@addtoreset{footnote}{page}
\makeatother
%
%「参考文献」を「References」にする
\renewcommand{\bibname}{References}
%「目次」を「Contents」にする
\renewcommand{\contentsname}{Contents}
%「第\CID{1624}章」を「Chapter x」にする
\renewcommand{\prechaptername}{Chapter }
\renewcommand{\postchaptername}{}

%表のキャプションを 表1.1ではなく Tabel1.1にする
\renewcommand{\tablename}{Table}
\renewcommand{\figurename}{Figure}


%% Appendix re-define
\renewcommand{\appendixname}{Appendix~}

%%目次にsabsectionを表示する
\setcounter{tocdepth}{2}

%
% 
\newcommand{\refeq}[1]{Eq.\,(\ref{#1})}
\renewcommand\vec\bm
\newcommand{\RR}{\vec{r}}
\newcommand{\PP}{\vec{p}}
\newcommand{\QQ}{\vec{q}}
\newcommand{\KK}{\vec{k}}
\newcommand{\TT}{\vec{t}}
\renewcommand{\SS}{\vec{s}}
\newcommand{\fermion}{\hat{\psi}}
\newcommand{\boson}{\hat{\phi}}
\newcommand{\BB}{\hat{b}}
\newcommand{\CC}{\hat{c}}
\newcommand{\momint}[1]{\frac{d^3{#1}}{(2\pi)^3}}
\newcommand{\sh}{\mathrm{Y}}
\newcommand{\schrodinger}{Schr\"{o}dinger }
\newcommand{\Tr}{\mathrm{Tr}}
\newcommand{\tr}{\mathrm{tr}}
\newcommand{\chrom}{\ce{Cr2O3}}
%
\newcommand{\debug}[1]{\textcolor{red}
{\textbf{[#1]}}}
\newcommand{\atom}[2][]{{}^{#1}\mathrm{#2}}
%
%
%

\begin{document}
% 数式の上下の余白を詰める
\setlength\abovedisplayskip{6.5pt}
\setlength\belowdisplayskip{6.5pt}

\chapter*{Abstract}
%\noindent % abstract text

Superconducting transition temperature ($T_{c}$) is one of the most important quantities of superconductors. 
Since the discovery of superconductors, theories and methods which can predict the $T_c$ from
first-principles are energerically studied. Recently, the density functional theory for 
superconductors (SCDFT) and relating numerical scheme has been developed. Applying the recently 
developed scheme, $T_c$s for various kinds of phonon-mediated conventional superconductors can be 
reproduced from first-principles in sufficient accuracy.

It is found that the pairing mechanism of recently discovered high-$T_c$ 
superconductors, e.g. Fe-based superconductors and cuprates, is not related to the electron-phonon 
interaction. Instead of electron-phonon interaction, the effect of spin fluctuations are suggested
as the possible pairing mechanism in these unconventional superconductors. According to this 
consideration, the extended version of SCDFT including the effect of spin fluctuations has been 
developed recently.

On the other hand, the effect of spin fluctuations is also crucial in phonon-mediated conventional
superconductors. In many metals in which electrons are nearly homogeneous, ferromagnetic fluctuations
are driven by the exchange effect and they suppress the singlet pairing. So far, this effect on 
$T_c$ of conventional superconductors has not been studied from first-principles.

In this thesis, we applied the recently developed extended SCDFT scheme to some elemental metals 
and studied the $T_c$ suppression due to spin fluctuations for these materials. As a result,
$T_c$ of V is reduced from 17.8K to 8.3K, and that of Nb is reduced from 11.1K to 8.9K by including
the effect of spin fluctuations and the agreement between experimental data is improved for both 
metals. On the other hand, $T_c$ of Al is not affected by spin fluctuations. These difference of
the effect of spin fluctuations is related to the magnitude of electronic localization.




\clearpage

\bibliographystyle{osajnl}
\bibliography{library_intro,library_method,library_result}

\end{document}
