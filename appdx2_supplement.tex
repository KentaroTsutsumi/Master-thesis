% まずはじめに \documentclass を指定する
\documentclass[uplatex]{jsbook}
%
% Packages
%
\usepackage[version=3]{mhchem}
\usepackage{geometry} % 余白の調整用
\usepackage{amsmath,amssymb}
\usepackage{booktabs} % 表組みのパッケージ
\usepackage{bm}
\usepackage[dvipdfmx]{graphicx}
\usepackage[subrefformat=parens]{subcaption}
\usepackage{ascmac}
\usepackage{braket} % ブラケット記法
\usepackage{cite} % [n-m]形式の引用を用いるため
%\usepackage{fancyhdr} % ヘッダ
\usepackage{feynmp} % Feynmanダイアグラムを書くため
\usepackage{color} % \debug 用に文字色をつける
\usepackage{setspace} % setstretchを使うため
\usepackage{newtxtext,newtxmath} % Times系フォントの使用

%\usepackage{abstract}
% 以下4つのパッケージはbibの著者名に出てきた欧文文字の文字化け対策として入れたが、
% 吉田は↓で上手くいく理由を良く理解していない。
% よく分からないパッケージは、それ無しで動くならコメントアウトしておくこと推奨。
\usepackage[uplatex,deluxe]{otf} %  \usepackage[prefernoncjk]{pxcjkcat} より先に読み込むべし(according to pxcjkcatのサイト)
\usepackage[prefernoncjk]{pxcjkcat}
\usepackage[T1]{fontenc} % T1エンコーディング, Bibliographyの著者用
%\usepackage[utf8]{inputenc}
%
% .bst ファイルは end.tex 内で指定することにしたから、次の行はコメントアウト
%\bibliographystyle{osajnlt}
%
% 行間と余白の調整
\setstretch{1.25}
\geometry{top=3truecm,bottom=3truecm,right=3truecm,left=3truecm}
%
% 見出しをセリフ・太字にする
\renewcommand{\headfont}{\bfseries}
%
% 脚注番号を記号に変える
\renewcommand{\thefootnote}{\fnsymbol{footnote}}
% 脚注記号を改ページでリセットする
\makeatletter
\@addtoreset{footnote}{page}
\makeatother
%
%「参考文献」を「References」にする
\renewcommand{\bibname}{References}
%「目次」を「Contents」にする
\renewcommand{\contentsname}{Contents}
%「第\CID{1624}章」を「Chapter x」にする
\renewcommand{\prechaptername}{Chapter }
\renewcommand{\postchaptername}{}

%表のキャプションを 表1.1ではなく Tabel1.1にする
\renewcommand{\tablename}{Table}
\renewcommand{\figurename}{Figure}


%% Appendix re-define
\renewcommand{\appendixname}{Appendix~}

%%目次にsabsectionを表示する
\setcounter{tocdepth}{2}

%
% 
\newcommand{\refeq}[1]{Eq.\,(\ref{#1})}
\renewcommand\vec\bm
\newcommand{\RR}{\vec{r}}
\newcommand{\PP}{\vec{p}}
\newcommand{\QQ}{\vec{q}}
\newcommand{\KK}{\vec{k}}
\newcommand{\TT}{\vec{t}}
\renewcommand{\SS}{\vec{s}}
\newcommand{\fermion}{\hat{\psi}}
\newcommand{\boson}{\hat{\phi}}
\newcommand{\BB}{\hat{b}}
\newcommand{\CC}{\hat{c}}
\newcommand{\momint}[1]{\frac{d^3{#1}}{(2\pi)^3}}
\newcommand{\sh}{\mathrm{Y}}
\newcommand{\schrodinger}{Schr\"{o}dinger }
\newcommand{\Tr}{\mathrm{Tr}}
\newcommand{\tr}{\mathrm{tr}}
\newcommand{\chrom}{\ce{Cr2O3}}
%
\newcommand{\debug}[1]{\textcolor{red}
{\textbf{[#1]}}}
\newcommand{\atom}[2][]{{}^{#1}\mathrm{#2}}
%
%
%
\begin{document}
% 数式の上下の余白を詰める
\setlength\abovedisplayskip{6.5pt}
\setlength\belowdisplayskip{6.5pt}
%
% Contents
%
\appendix
\chapter{Numerical method to execute the Matsubara summation}
\label{appdx:Msum}

In this appendix, we describe how to execute the Matsubara summation in (\ref{eq:Keldyn}) 
following Ref.\cite{RAphD}. We also explain the difference of the numerical scheme between in case of
Coulomb interaction (\ref{eq:Keldyn}) and the effect of spin fluctuations (\ref{eq:KSFdef}).

The quantity which we have to evaluate is 
%
\begin{equation}
	\begin{split}
        {\mathcal K}^{\rm el, dyn}_{n\bm k,n'\bm k'} & =
	\lim_{\{\Delta_{n\bm k}\} \to 0}
	\frac{1}{\tanh[(\beta/2)E_{n\bm k}]}\frac{1}{\tanh[(\beta/2)E_{n'\bm k'}]} \\
	&\quad \times
	\frac{1}{\beta^2}
	\sum_{\tilde{\omega}_1\tilde{\omega}_2}
	F_{n\bm k}({\rm i}\tilde\omega_1)F_{n'\bm k'}({\rm i}\tilde\omega_2)
	W_{n\bm kn'\bm k'}[{\rm i}(\tilde\omega_1 - \tilde\omega_2)],
	\label{eq:Keldyn}
\end{split}
\end{equation}
%
where the function $F$ is defined as
%
\begin{equation}
F_{n\bm k}({\rm i}\tilde\omega) = 
\frac{1}{ {\rm i} \tilde\omega + E_{n\bm k}}-\frac{1}{ {\rm i} \tilde\omega - E_{n\bm k}},
	\label{eq:Ffunc}
\end{equation}
%
and $\tilde\omega$ means the fermionic Matsubara frequency and
$W_{n\bm kn'\bm k'}$ is the matrix element of the dynamical Coulomb interaction.
We explain the scheme for the summation with respect to 
$\Lambda^{\rm SF}_{n\bm k n'\bm k'}({\rm i}\tilde{\omega})$ later.

At first, we transform the variable as $\tilde{\nu} \equiv \tilde{\omega}_1 - \tilde{\omega}_2$.
Furthermore, we separate the static term $W_{n\bm k n' \bm k'}(0)$ from the finite-frequency term.
The reason why this separation is applied will be described later.
By using the relation $W_{n \bm k n' \bm k'}(-{\rm i}\omega) = W_{n\bm k n' \bm k'}({\rm i}\omega)$
and the formula (\ref{eq:sumform1}),
the Matsubara summation with respect to $\tilde{\omega}_1$ can be analytically carried out as follows
%
\begin{equation}
	\begin{split}
	&\frac{1}{\beta^2}
	\sum_{\tilde{\omega}_1\tilde{\omega}_2}
	F_{n\bm k}({\rm i}\tilde\omega_1)F_{n'\bm k'}({\rm i}\tilde\omega_2)
	W_{n\bm kn'\bm k'}[{\rm i}(\tilde\omega_1 - \tilde\omega_2)] \\
	&\quad =
	\tanh[(\beta/2)E_{n\bm k}]\tanh[(\beta/2)E_{n' \bm k'}]W_{n\bm k n' \bm k'}(0) \\
	& \quad \quad +
	\frac{2}{\beta}\sum_{\tilde{\nu}}
	[W_{n\bm k n' \bm k'}({\rm i}\tilde{\nu}) - W_{n\bm k n' \bm k'}(0)]
	\left[ 
		\frac{f_\beta(E_{n'\bm k'}) - f_\beta(E_{n\bm k})}{ {\rm i}\tilde{\nu} - (E_{n\bm k}-E_{n'\bm k'})}
		- \frac{f_\beta(E_{n'\bm k'}) - f_\beta(-E_{n\bm k})}{ {\rm i}\tilde{\nu} - (-E_{n\bm k}-E_{n'\bm k'})}
	\right] \\
	& \quad =
	\tanh[(\beta/2)E_{n\bm k}]\tanh[(\beta/2)E_{n' \bm k'}]W_{n\bm k n' \bm k'}(0) \\
	& \quad \quad +
	\frac{4}{\beta}\sum_{\tilde{\nu} > 0}
	[W_{n\bm k n' \bm k'}({\rm i}\tilde{\nu}) - W_{n\bm k n' \bm k'}(0)] \\
	& \quad \quad \quad \times
	\left[ 
		\frac{[f_\beta(E_{n'\bm k'}) - f_\beta(E_{n\bm k})](E_{n'\bm k'} - E_{n\bm k})}
		{ \tilde{\nu}^2 + (E_{n\bm k}-E_{n'\bm k'})^2}
		-\frac{[f_\beta(E_{n'\bm k'}) - f_\beta(-E_{n\bm k})](E_{n'\bm k'} + E_{n\bm k})}
		{ \tilde{\nu}^2 + (E_{n\bm k}+E_{n'\bm k'})^2}
	\right], 
	\label{eq:nutrans}
\end{split}
\end{equation}
%
where $f_{\beta}$ is the fermi distribution function. Based on this separation, the kernel
${\mathcal K}^{\rm el}$ can be written as ${\mathcal K}^{\rm el} = {\mathcal K}^{\rm el,stat} +
\Delta{\mathcal K}^{\rm el,dyn}$ where
%
\begin{equation}
	\begin{split}
	\Delta{\mathcal K}^{\rm el,dyn}_{n\bm k, n'\bm k'} & =
	\frac{1}{\tanh[(\beta/2)\xi_{n\bm k}]\tanh[(\beta/2)\xi_{n'\bm k'}]} \\
	&\quad \times
	[ 
		\{f_{\beta}(\xi_{n'\bm k'}) - f_{\beta}(\xi_{n\bm k})\}{\rm sgn}(\xi_{n'\bm k'}-\xi_{n\bm k})
		L_{n\bm k n' \bm k'}(|\xi_{n'\bm k'}-\xi_{n\bm k}|) \\
	&\quad \quad -
	\{f_{\beta}(\xi_{n'\bm k'}) - f_{\beta}(-\xi_{n\bm k})\}{\rm sgn}(\xi_{n'\bm k'}+\xi_{n\bm k})
		L_{n\bm k n' \bm k'}(|\xi_{n'\bm k'}+\xi_{n\bm k}|) 
	],
	\label{eq:deltaKeldyn}
\end{split}
\end{equation}
%
where the function $L_{n\bm k n \bm k'}$ is defined as 
%
\begin{equation}
	L_{n\bm k n' \bm k'}(x) = \frac{4}{\beta}\sum_{\tilde{\nu}>0}
	\left[ 
		W_{n\bm k n' \bm k'}({\rm i}\tilde{\nu}) - W_{n\bm k n' \bm k'}(0)
	\right]
	\frac{x}{\tilde{\nu}^2 + x^2}.
	\label{eq:Lfunc}
\end{equation}
%

Now the remaining problem is how to evaluate the bosonic Matsubara summation.
In order to construct an algorithm to calculate this, we use the Euler-Maclaurin formula



\include{end}

