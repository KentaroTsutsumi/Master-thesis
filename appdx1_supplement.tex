% まずはじめに \documentclass を指定する
\documentclass[uplatex]{jsbook}
%
% Packages
%
\usepackage[version=3]{mhchem}
\usepackage{geometry} % 余白の調整用
\usepackage{amsmath,amssymb}
\usepackage{booktabs} % 表組みのパッケージ
\usepackage{bm}
\usepackage[dvipdfmx]{graphicx}
\usepackage[subrefformat=parens]{subcaption}
\usepackage{ascmac}
\usepackage{braket} % ブラケット記法
\usepackage{cite} % [n-m]形式の引用を用いるため
%\usepackage{fancyhdr} % ヘッダ
\usepackage{feynmp} % Feynmanダイアグラムを書くため
\usepackage{color} % \debug 用に文字色をつける
\usepackage{setspace} % setstretchを使うため
\usepackage{newtxtext,newtxmath} % Times系フォントの使用

%\usepackage{abstract}
% 以下4つのパッケージはbibの著者名に出てきた欧文文字の文字化け対策として入れたが、
% 吉田は↓で上手くいく理由を良く理解していない。
% よく分からないパッケージは、それ無しで動くならコメントアウトしておくこと推奨。
\usepackage[uplatex,deluxe]{otf} %  \usepackage[prefernoncjk]{pxcjkcat} より先に読み込むべし(according to pxcjkcatのサイト)
\usepackage[prefernoncjk]{pxcjkcat}
\usepackage[T1]{fontenc} % T1エンコーディング, Bibliographyの著者用
%\usepackage[utf8]{inputenc}
%
% .bst ファイルは end.tex 内で指定することにしたから、次の行はコメントアウト
%\bibliographystyle{osajnlt}
%
% 行間と余白の調整
\setstretch{1.25}
\geometry{top=3truecm,bottom=3truecm,right=3truecm,left=3truecm}
%
% 見出しをセリフ・太字にする
\renewcommand{\headfont}{\bfseries}
%
% 脚注番号を記号に変える
\renewcommand{\thefootnote}{\fnsymbol{footnote}}
% 脚注記号を改ページでリセットする
\makeatletter
\@addtoreset{footnote}{page}
\makeatother
%
%「参考文献」を「References」にする
\renewcommand{\bibname}{References}
%「目次」を「Contents」にする
\renewcommand{\contentsname}{Contents}
%「第\CID{1624}章」を「Chapter x」にする
\renewcommand{\prechaptername}{Chapter }
\renewcommand{\postchaptername}{}

%表のキャプションを 表1.1ではなく Tabel1.1にする
\renewcommand{\tablename}{Table}
\renewcommand{\figurename}{Figure}


%% Appendix re-define
\renewcommand{\appendixname}{Appendix~}

%%目次にsabsectionを表示する
\setcounter{tocdepth}{2}

%
% 
\newcommand{\refeq}[1]{Eq.\,(\ref{#1})}
\renewcommand\vec\bm
\newcommand{\RR}{\vec{r}}
\newcommand{\PP}{\vec{p}}
\newcommand{\QQ}{\vec{q}}
\newcommand{\KK}{\vec{k}}
\newcommand{\TT}{\vec{t}}
\renewcommand{\SS}{\vec{s}}
\newcommand{\fermion}{\hat{\psi}}
\newcommand{\boson}{\hat{\phi}}
\newcommand{\BB}{\hat{b}}
\newcommand{\CC}{\hat{c}}
\newcommand{\momint}[1]{\frac{d^3{#1}}{(2\pi)^3}}
\newcommand{\sh}{\mathrm{Y}}
\newcommand{\schrodinger}{Schr\"{o}dinger }
\newcommand{\Tr}{\mathrm{Tr}}
\newcommand{\tr}{\mathrm{tr}}
\newcommand{\chrom}{\ce{Cr2O3}}
%
\newcommand{\debug}[1]{\textcolor{red}
{\textbf{[#1]}}}
\newcommand{\atom}[2][]{{}^{#1}\mathrm{#2}}
%
%
%

\begin{document}
% 数式の上下の余白を詰める
\setlength\abovedisplayskip{6.5pt}
\setlength\belowdisplayskip{6.5pt}
%
% Contents
%
\appendix
\chapter{Derivation of the SCDFT gap equation}
\label{appdx:gapeq}
In this appendix, we review the detail of the derivation of the SCDFT gap equation
(\ref{eq:aftergapeq}) from the Sham-Schl\"{u}ter connection according to Ref.\cite{MarquesphD}.

We start from the following self-energy
%
\begin{equation}
	\begin{split}
	\bar{\Sigma}^{\rm SS} &= \bar{\Sigma}^{\rm GW} + \bar{\Sigma}^{\rm SF} + \bar{\Sigma}^{\rm ph}
	-\delta(\tau - \tau') 
	\begin{pmatrix}
		 v_{\rm xc}(\bm r)\delta(\bm r - \bm r')  &   \Delta^{\rm xc}(\bm r, \bm r')   \\
		 \Delta^{\rm xc\ast}(\bm r, \bm r')  & - v_{\rm xc}(\bm r)\delta(\bm r - \bm r')
	\end{pmatrix} \\
	& \equiv \bar{\Sigma}^{\rm xc} - \delta(\tau - \tau')
	\begin{pmatrix}
		 v_{\rm xc}(\bm r)\delta(\bm r - \bm r')  &   \Delta^{\rm xc}(\bm r, \bm r')   \\
		 \Delta^{\rm xc\ast}(\bm r, \bm r')  & - v_{\rm xc}(\bm r)\delta(\bm r - \bm r')
	\end{pmatrix} \\
	\label{eq:appSSself}
\end{split}
\end{equation}
%
\begin{equation}
	\bar{\Sigma}^{\rm GW}(\bm r \tau, \bm r' \tau') = \tau^z v^{\rm screen}(\bm r \tau, \bm r' \tau')
	\begin{pmatrix}
		0  &   F^{\rm KS}(\bm r \tau, \bm r' \tau')   \\
		F^{\rm KS\dag}(\bm r \tau, \bm r' \tau')  & 0
	\end{pmatrix},
	\label{eq:GWself}
\end{equation}
%
\begin{equation}
	\bar{\Sigma}^{\rm ph}(\bm r \tau, \bm r' \tau') = \tau^z \sum_{\lambda \bm q}
	V^{\rm BO}_{\lambda \bm q}(\bm r)V^{\rm BO}_{\lambda -\bm q}(\bm r') D^{\rm s}_{\lambda\bm q}(\tau, \tau')
	\begin{pmatrix}
		G^{\rm KS}(\bm r \tau, \bm r' \tau')  &   F^{\rm KS}(\bm r \tau, \bm r' \tau')   \\
		F^{\rm KS\dag}(\bm r \tau, \bm r' \tau')  & G^{\rm KS\dag}(\bm r' \tau', \bm r \tau)
	\end{pmatrix},
	\label{eq:phself}
\end{equation}
%
\begin{equation}
	\bar{\Sigma}^{\rm SF}(\bm r \tau, \bm r' \tau') = 3\tau^z 
	\Lambda^{\rm SF}(\bm r \tau, \bm r' \tau')
	\begin{pmatrix}
		-G^{\rm KS}(\bm r \tau, \bm r' \tau')  &   F^{\rm KS}(\bm r \tau, \bm r' \tau')   \\
		F^{\rm KS\dag}(\bm r \tau, \bm r' \tau')  & -G^{\rm KS\dag}(\bm r' \tau', \bm r \tau)
	\end{pmatrix},
	\label{eq:SFself}
\end{equation}
%
where $\tau^z$ is the third Pauli matrix. 
With the above self-energy, the Dyson's equation of Green's equation is written as 
%
\begin{equation}
	\bar{G}(\bm r \tau, \bm r' \tau') = \bar{G}^{\rm KS}(\bm r \tau, \bm r' \tau') +
	\int d^{3}(r_1 r_2)\int d(\tau_1 \tau_2)\bar{G}^{\rm KS}(\bm r \tau, \bm r_1 \tau_1)
	\bar{\Sigma}^{\rm SS}(\bm r_1 \tau, \bm r_2 \tau_2)\bar{G}(\bm r_2 \tau_2, \bm r' \tau'),
	\label{eq:GDysoneq}
\end{equation}
%
where $\bar{G}$ indicates the full Green's function. 

If the Hamiltonian is time-independent, the
Green's functions depend on the difference of time $\tau - \tau'$. Therefore it is convenient
to carry out the Fourier transformation with respect to time as follows
%
\begin{equation}
	G^{\rm KS}(\bm r \tau, \bm r' \tau') = \frac{1}{\beta} \sum_{\omega_n}
	e^{-{\rm i}\omega_n(\tau-\tau')} G^{\rm KS}(\bm r,\bm r', \omega_n),
	\label{eq:GFourier1}
\end{equation}
%
\begin{equation}
	G^{\rm KS}(\bm r, \bm r', \omega_n) = \int_{0}^{\beta} d(\tau-\tau')
	e^{{\rm i}\omega_n(\tau-\tau')} G^{\rm KS}(\bm r \tau,\bm r' \tau'),
	\label{eq:GFourier2}
\end{equation}
%
where $\omega_n$ is the odd Matsubara frequency. It is also convenient to introduce the Fourier
transformation with respect to space as follows
%
\begin{equation}
	G^{\rm KS}(\bm r, \bm r', \omega_n) = \sum_{ k}\varphi_{\bm k}(\bm r)
	G^{\rm KS}( k, \omega_n)\varphi_{ k}(\bm r'),
\label{eq:GFourier3}
\end{equation}
%
\begin{equation}
	G^{\rm KS}( k, \omega_n) = \int d^3(rr') \varphi^{\ast}_{ k}(\bm r)
	G^{\rm KS}(\bm r,\bm r', \omega_n)
	\varphi_{ k}(\bm r'),
	\label{eq:GFourier4}
\end{equation}
%
where $k=\{ n, \bm k \}$ is the set of band index and crystal momentum and 
$\varphi_{ k}(\bm r)$ is the one-particle wavefunction. After these transformations,
the Green's functions can be written as
%
\begin{equation}
	G^{\rm KS}( k, \omega_n) = \frac{|u_{ k}|^2}{ {\rm i}\omega_n - E_{ k}}
	- \frac{|v_{ k}|^2}{ {\rm i}\omega_n + E_{ k}},
	\label{eq:Gkomega}
\end{equation}
%
\begin{equation}
	F^{\rm KS}( k, \omega_n) = - u_{ k}v_{ k}^{\ast} \left( 
	\frac{1}{ {\rm i}\omega_n + E_{ k}} -\frac{1}{ {\rm i}\omega_n - E_{ k}}
	\right),
	\label{eq:Fkomega}
\end{equation}
%
where $E_{ k} $ is defined in (\ref{eq:KS-energy}), $u_{ k}$ and $v_{ k}$ can be obtained
through the Kohn-Sham BdG equation (see (\ref{eq:KS-BdG-par}) and (\ref{eq:KS-BdG-hole})) 
and written as
%
\begin{equation}
	u_{ k} = \frac{1}{\sqrt{2}} {\rm sgn}(\tilde{E}_{ k})e^{ {\rm i}\phi_{ k}}
	\sqrt{1 + \frac{\xi_{ k}}{\tilde{E}_{ k}}}
	\label{eq:uk}
\end{equation}
%
\begin{equation}
	v_{ k} = \frac{1}{\sqrt{2}} \sqrt{1 - \frac{\xi_{ k}}{\tilde{E}_{ k}}},
	\label{eq:vk}
\end{equation}
%
\begin{equation}
	e^{ {\rm i}\phi_{ k}} = \frac{\Delta_{ k}}{|\Delta_{ k}|}.
	\label{eq:phase}
\end{equation}
%

With these preparations, we go back to the main story line. After the Fourier transformation into frequency space,
the Sham-Schl\"{u}ter connection can be written as
%
\begin{equation}
	0 = \lim_{\bm r \to \bm r'} \frac{2}{\beta} \sum_{\omega_n}
	e^{ {\rm i}\omega_n 0^{+}} 
	\int d^3(r_1 r_2) \left[ \bar{G}^{\rm KS}(\bm r_1, \bm r_2, \omega_n)
	\bar{\Sigma }^{\rm SS}(\bm r_1, \bm r_2, \omega_n) \bar{G}(\bm r_2, \bm r', \omega_n)
	\right]_{11},
	\label{eq:AppSSrho}
\end{equation}
%
\begin{equation}
	0 = - \frac{1}{\beta} \sum_{\omega_n}
	e^{ {\rm i}\omega_n 0^{+}} 
	\int d^3(r_1 r_2) \left[ \bar{G}^{\rm KS}(\bm r_1, \bm r_2, -\omega_n)
	\bar{\Sigma }^{\rm SS}(\bm r_1, \bm r_2, -\omega_n) \bar{G}(\bm r_2, \bm r', -\omega_n)
	\right]_{12}.
	\label{eq:AppSSanom}
\end{equation}
%
These are the generalized Sham-Schl\"{u}ter equations for SCDFT.

We describe more technical and complex derivation in the following.
We decompose $\bar{\Sigma}^{\rm SS} $ into two terms like in (\ref{eq:appSSself}) and rewrite the 
above two equations as follows:
%
\begin{equation}
	\begin{split}
	& \frac{2}{\beta} \sum_{\omega_n}
	\int d^3(r_1 r_2) \left[ \bar{G}^{\rm KS}(\bm r, \bm r_1, \omega_n)
	\bar{\Sigma }^{\rm xc}(\bm r_1, \bm r_2, \omega_n) \bar{G}(\bm r_2, \bm r, \omega_n)
	\right]_{11} \\
	&\quad =
	\frac{2}{\beta}\sum_{\omega_n} 
	v_{\rm xc}(\bm r_1) [G^{\rm KS}(\bm r,\bm r_1,\omega_n)G(\bm r_1, \bm r, \omega_n)+
	F^{\rm KS}(\bm r, \bm r_1, \omega_n)F^{\dag}(\bm r_1, \bm r, \omega_n)] \\
	&\quad \quad + 
	\frac{2}{\beta}\sum_{\omega_n}
	\int d^3r_1 d^3r_2 
	[\Delta_{\rm xc}^{\ast}(\bm r_1, \bm r_2)
	F^{\rm KS}(\bm r,\bm r_1, \omega_n)G(\bm r_2, \bm r, \omega_n) 
	- \Delta_{\rm xc}(\bm r_1, \bm r_2)
	G^{\rm KS}(\bm r,\bm r_1, \omega_n)F^{\dag}(\bm r_2, \bm r, \omega_n) 
	],
	\label{eq:11comp1}
\end{split}
\end{equation}
%
\begin{equation}
	\begin{split}
	& \frac{1}{\beta} \sum_{\omega_n}
	\int d^3(r_1 r_2) \left[ \bar{G}^{\rm KS}(\bm r, \bm r_1, -\omega_n)
	\bar{\Sigma }^{\rm xc}(\bm r_1, \bm r_2, -\omega_n) \bar{G}(\bm r_2, \bm r', -\omega_n)
	\right]_{12} \\
	&\quad =
	\frac{1}{\beta}\sum_{\omega_n} 
	v_{\rm xc}(\bm r_1) [G^{\rm KS}(\bm r,\bm r_1,-\omega_n)F(\bm r_1, \bm r', -\omega_n)+
	F^{\rm KS}(\bm r, \bm r_1, -\omega_n)G(\bm r', \bm r_1, \omega_n)] \\
	&\quad \quad + 
	\frac{1}{\beta}\sum_{\omega_n}
	\int d^3r_1 d^3r_2 
	[\Delta_{\rm xc}^{\ast}(\bm r_1, \bm r_2)
	F^{\rm KS}(\bm r,\bm r_1, -\omega_n)F(\bm r_2, \bm r', -\omega_n) 
	- \Delta_{\rm xc}(\bm r_1, \bm r_2)
	G^{\rm KS}(\bm r,\bm r_1, -\omega_n)G(\bm r', \bm r_2, \omega_n) 
	].
	\label{eq:12comp1}
\end{split}
\end{equation}
%
where we dropped the factor $e^{ {\rm i}\omega_n0^{+}}$ for simplicity. 
These two equations determines the exchange-correlation(xc) potentials together.
We now carry out the Fourier transformation into $k$-space and write the matrix elements explicitly
as follows:
%
\begin{equation}
	\begin{split}
	& \frac{1}{\beta} \sum_{k,\omega_n} [\Sigma^{\rm xc}_{11}(k,k,\omega_n)
	  G^{\rm KS}(k,\omega_n)G(k,\omega_n) + \Sigma^{\rm xc}_{21}(k,k,\omega_n)
	  F^{\rm KS}(k,\omega_n)G(k,\omega_n) \\
  	&\quad \quad \quad - \Sigma^{\rm xc}_{12}(k,k,\omega_n)G^{\rm KS}(k,\omega_n)F^{\dag}(k,\omega_n)
	- \Sigma^{\rm xc}_{22}(k,k,\omega_n)F^{\rm KS}(k,\omega_n)F^{\dag}(k,\omega_n)] \\
	& = \frac{1}{\beta}\sum_{k, \omega_n} v_{\rm xc}(k,k)
	[G^{\rm KS}(k,\omega_n)G(k,\omega_n) + F^{\rm KS}(k,\omega_n)F^{\dag}(k,\omega_n)] \\
	&\quad + \frac{1}{\beta} \sum_{k,\omega_n}[\Delta^{\rm xc\ast}_{k}
		F^{\rm KS}(k, \omega_n)G(k, \omega_n) - \Delta^{\rm xc}_{k}G^{\rm KS}(k,\omega_n)
	F^{\dag}(k,\omega_n)],
	\label{eq:11comp2}
\end{split}
\end{equation}
%
\begin{equation}
	\begin{split}
	& \frac{1}{\beta} \sum_{\omega_n} [\Sigma^{\rm xc}_{11}(k,k',-\omega_n)
	  G^{\rm KS}(k,-\omega_n)F(k',-\omega_n) + \Sigma^{\rm xc}_{21}(k,k',-\omega_n)
	  F^{\rm KS}(k,-\omega_n)G(k',\omega_n) \\
  	&\quad \quad \quad - \Sigma^{\rm xc}_{12}(k,k',\omega_n)G^{\rm KS}(k,-\omega_n)G(k',\omega_n)
	- \Sigma^{\rm xc}_{22}(k,k',-\omega_n)F^{\rm KS}(k,-\omega_n)G(k',\omega_n)] \\
	& = \frac{1}{\beta}\sum_{\omega_n} v_{\rm xc}(k,k')
	[G^{\rm KS}(k,-\omega_n)F(k',-\omega_n) + F^{\rm KS}(k,-\omega_n)G(k',\omega_n)] \\
	&\quad + \frac{1}{\beta} \sum_{\omega_n}\delta_{kk'}[\Delta^{\rm xc\ast}_{k}
		F^{\rm KS}(k, -\omega_n)F(k', -\omega_n) - \Delta^{\rm xc}_{k}G^{\rm KS}(k,-\omega_n)
	G(k',\omega_n)],
	\label{eq:12comp2}
\end{split}
\end{equation}
Here we defined the $k$-space quantities as follows:
%
\begin{equation}
	v_{\rm xc}(k,k') = \int d^3r \varphi^{\ast}_{k}(\bm r)v_{\rm xc}(\bm r)\varphi_{k'}(\bm r),
	\label{eq:vxckk}
\end{equation}
%
\begin{equation}
	\Sigma(k,k',\omega_n) = \int d^3r d^3r' \varphi^{\ast}_{k}(\bm r)
	\Sigma(\bm r,\bm r',\omega_n)\varphi_{k'}(\bm r').
	\label{eq:Sigmakk}
\end{equation}
%
In the following, we apply an approximation because these two coupled equations are difficult 
to solve directly. At first we set $k' = k$ in (\ref{eq:12comp2}), and we carry out an 
approximation that we neglect the $k$ dependence of the xc potential:
%
\begin{equation}
	v_{\rm xc}(k,k) \approx v_{\rm xc}.
	\label{eq:vxcapprox}
\end{equation}
%
This approximation is valid if the system we consider is nearly homogeneous and the xc potential is
nearly constant everywhere in the unit cell. For this reason this approximation is not appropriate
in case of highly inhomogeneous materials such as high-$T_c$ layered superconductors.
Within this approximation, equation (\ref{eq:12comp2}) can be solved with respect to $v_{\rm xc}$.
Substituting the form of $v_{\rm xc}$ into (\ref{eq:11comp2}) yields
%
\begin{equation}
\begin{split}
	& \frac{1}{\beta} \sum_{k',\omega_n} [\Sigma^{\rm xc}_{11}(k',k',\omega_n)
	  G^{\rm KS}(k',\omega_n)G(k',\omega_n) + \Sigma^{\rm xc}_{21}(k',k',\omega_n)
	  F^{\rm KS}(k',\omega_n)G(k',\omega_n) \\
  	&\quad \quad \quad - \Sigma^{\rm xc}_{12}(k',k',\omega_n)G^{\rm KS}(k',\omega_n)F^{\dag}(k',\omega_n)
	- \Sigma^{\rm xc}_{22}(k',k',\omega_n)F^{\rm KS}(k',\omega_n)F^{\dag}(k',\omega_n) \\
	&\quad \quad \quad -\Delta^{\rm xc \ast}_{k'}F^{\rm KS}(k',\omega_n)G(k,\omega_n) + 
	\Delta^{\rm xc}_{k'}G^{\rm KS}(k',\omega_n)F^{\dag}(k',\omega_n)] \\
	&\quad \quad \times \frac{1}{\beta}\sum_{\omega_m} [G^{\rm KS}(k,\omega_m)F(k,\omega_m) + 
	F^{\rm KS}(k,-\omega_m)G(k,\omega_m)] \\
	& = 
	 \frac{1}{\beta} \sum_{\omega_n} [\Sigma^{\rm xc}_{11}(k,k',-\omega_n)
	  G^{\rm KS}(k,-\omega_n)F(k',-\omega_n) + \Sigma^{\rm xc}_{21}(k,k',-\omega_n)
	  F^{\rm KS}(k,-\omega_n)G(k',\omega_n) \\
  	&\quad \quad \quad \quad- \Sigma^{\rm xc}_{12}(k,k',\omega_n)G^{\rm KS}(k,-\omega_n)G(k',\omega_n)
	- \Sigma^{\rm xc}_{22}(k,k',-\omega_n)F^{\rm KS}(k,-\omega_n)G(k',\omega_n) \\
	&\quad \quad \quad \quad-\Delta^{\rm xc \ast}_{k}F^{\rm KS}(k,-\omega_n)F(k,-\omega_n)
	+ \Delta^{\rm KS}_{k}G^{\rm KS}(k,\omega_n)G(k,\omega_n) ] \\
	& \quad \quad \times 
	\frac{1}{\beta} \sum_{k',\omega_m}[ G^{\rm KS}(k',\omega_m)G(k',\omega_m) + 
	F^{\rm KS}(k', \omega_m)F^{\dag}(k',\omega_m)].
	\label{eq:vxcvanish}
\end{split}
\end{equation}
%
The above equation is also complex but can be simplified by linealization.
We are mainly interested in the calculation of the transition temperature rather than the full temperature 
dependence of $\Delta^{\rm xc}$. Therefore we neglect the higher-order term of $F, F^{\dag}$ and $\Delta^{\rm xc}$.
Then (\ref{eq:vxcvanish}) is simplified as follows:
%
\begin{equation}
	\begin{split}
		&\Delta^{\rm xc}_{k}\frac{1}{\beta}\sum_{\omega_n}G^{\rm KS}(k,-\omega_n)G(k,\omega_n)
		\frac{1}{\beta}\sum_{k',\omega_m}G^{\rm KS}(k',\omega_m)G(k',\omega_m) \\
		& \quad =
		\frac{1}{\beta}\sum_{\omega_n} [-\Sigma^{\rm xc}_{11}(k,k,-\omega_n)G^{\rm KS}(k,-\omega_n)F(k,-\omega_n)
		+\Sigma^{\rm xc}_{12}(k,k,-\omega_n)G^{\rm KS}(k,-\omega_n)G(k,\omega_n) \\
		&\quad \quad \quad \quad \quad - \Sigma^{\rm xc}_{11}(k,k,\omega_n)F^{\rm KS}(k,-\omega_n)G(k,\omega_n)]
		\times \frac{1}{\beta}\sum_{k',\omega_m} G^{\rm KS}(k',\omega_m)G(k', \omega_m) \\
		& \quad \quad +
		\frac{1}{\beta}\sum_{k',\omega_n}\Sigma^{\rm xc}_{11}(k',k',\omega_n)G^{\rm KS}(k',\omega_n)
		G(k,\omega_n)\frac{1}{\beta}\sum_{\omega_m}
		[G^{\rm KS}(k,-\omega_m)F(k,-\omega_m) + F^{\rm KS}(k,-\omega_m)G(k,\omega_m)].
	\end{split}
	\label{eq:Deltaxceq}
\end{equation}
%
Now we approximate the full Green's function by the Kohn-Sham Green's functions.
Then we can carry out some of the Matsubara summation analytically making use of the following formula
%
\begin{equation}
	\frac{1}{\beta} \sum_{\omega_n} \frac{1}{ {\rm i}\omega_n - x} = 
	\frac{1}{e^{\beta x}+1},
	\label{eq:sumform1}
\end{equation}
%
\begin{equation}
	\frac{1}{\beta} \sum_{\omega_n} \frac{1}{ ({\rm i}\omega_n - x)^2} = 
	\frac{-\beta e^{\beta x}}{(e^{\beta x}+1)^2}.
	\label{eq:sumform2}
\end{equation}
%
In the following we summarize the calculation results: 
%
\begin{equation}
	\frac{1}{\beta}\sum_{\omega_n}
	G^{\rm KS}(k,\omega_n)G^{\rm KS}(k,-\omega_n) 
	= \frac{1}{2E_{k}}\tanh\left( \frac{\beta}{2}E_{k} \right),
	\label{eq:GGsum1}
\end{equation}
%
\begin{equation}
	\frac{1}{\beta}\sum_{\omega_n}
	G^{\rm KS}(k,\omega_n)G^{\rm KS}(k,\omega_n) 
	= - \frac{1}{2} \frac{\beta/2}{\cosh^2\left( \frac{\beta}{2}\xi_{k} \right)},
	\label{eq:GGsum2}
\end{equation}
%
\begin{equation}
	\frac{1}{\beta}\sum_{\omega_n}
	[G^{\rm KS}(k,\omega_n)F^{\rm KS}(k,\omega_n) + F^{\rm KS}(k,\omega_n)G^{\rm KS}(k,-\omega_n)]
	= -\frac{\Delta^{\rm xc}_{k}}{2\xi_{k}} 
	\left[ \frac{\beta/2}{\cosh^2\left( \frac{\beta}{2}\xi_{k} \right)} -
	\frac{\tanh\left( \frac{\beta}{2}\xi_{k} \right)}{\xi_{k}} \right],
	\label{eq:GFFGsum}
\end{equation}
%
while the terms containing the self-energy originating from the Coulomb interaction are written as:
%
\begin{equation}
	\frac{1}{\beta}\sum_{\omega_n} G^{\rm KS}(k,\omega_n)\Sigma^{\rm GW}_{11}(k,k,\omega_n)G^{\rm KS}(k,\omega_n)
	= 0,
	\label{eq:GW11sum1}
\end{equation}
%
\begin{equation}
	\frac{1}{\beta}\sum_{\omega_n} G^{\rm KS}(k,\omega_n)\Sigma^{\rm GW}_{11}(k,k,\omega_n)F^{\rm KS}(k,\omega_n)
	= 0,
	\label{eq:GW11sum2}
\end{equation}
%
\begin{equation}
	\begin{split}
	\frac{1}{\beta}\sum_{k''} G^{\rm KS}(k,\omega_n)\Sigma^{\rm GW}_{12}(k,k,\omega_n)G^{\rm KS}(k,-\omega_n)
	& = -\frac{1}{4\beta^2}\sum_{k',\omega_n,\omega_m}
	\frac{\Delta^{\rm xc}_{k'}}{E_{k}E_{k'}}v^{\rm screen}(k,k',\omega_n - \omega_m) \\
	&\quad \times \left( \frac{1}{ {\rm i}\omega_n + E_{k}} - \frac{1}{ {\rm i}\omega_n - E_{k}} \right)
	\left( \frac{1}{ {\rm i}\omega_m + E_{k'}} - \frac{1}{ {\rm i}\omega_m - E_{k'}} \right).
	\label{eq:GW12sum}
\end{split}
\end{equation}
%
Terms of the self-energy stemming from the electron-phonon interaction are written as:
%
\begin{equation}
	\frac{1}{\beta} \sum_{\omega_n}G^{\rm KS}(k,\omega_n)\Sigma^{\rm ph}_{11}(k,k,\omega_n)G^{\rm KS}(k,\omega_n)
	= \sum_{k',\lambda,q} |g^{kk'}_{\lambda,q}|^2 I'(\xi_k,\xi_{k'},\Omega_{\lambda q}),
	\label{eq:ph11sum1}
\end{equation}
%
\begin{equation}
	\begin{split}
	&\frac{1}{\beta} \sum_{\omega_n}G^{\rm KS}(k,\omega_n)\Sigma^{\rm ph}_{12}(k,k,\omega_n)G^{\rm KS}(k,-\omega_n)\\
	& \quad =
	-\frac{1}{2}\sum_{k',\lambda,q}\frac{\Delta^{\rm xc}_{k'}}{E_{k}E_{k'}}
	|g^{kk'}_{\lambda,q}|^2 [I(\xi_k,\xi_{k'},\Omega_{\lambda q})-I(\xi_k,-\xi_{k'},\Omega_{\lambda q})],
	\end{split}
	\label{eq:ph12sum}
\end{equation}
%
\begin{equation}
	\begin{split}
	&\frac{1}{\beta} \sum_{\omega_n}G^{\rm KS}(k,\omega_n)\Sigma^{\rm ph}_{11}(k,k,\omega_n)F^{\rm KS}(k,\omega_n)
	=\frac{1}{\beta} \sum_{\omega_n}G^{\rm KS}(k,-\omega_n)\Sigma^{\rm ph}_{11}(k,k,-\omega_n)F^{\rm KS}(k,\omega_n)\\
	& \quad = -
	\frac{\Delta^{\rm xc}_{k}}{2\xi_{k}} \sum_{k',\lambda,q} |g^{kk'}_{\lambda,q}|^2
	\left\{ I'(\xi_k,\xi_{k'},\Omega_{\lambda,q})-\frac{1}{2\xi_k}
	\left[ I(\xi_k,\xi_{k'},\Omega_{\lambda,q}) - I(\xi_k,-\xi_{k'},\Omega_{\lambda,q}) \right]\right\}.
	\label{eq:ph11sum2}
	\end{split}
\end{equation}
%
Finally, terms including the self-energy originating from spin fluctuations is written as:
%
\begin{equation}
	\frac{1}{\beta} \sum_{\omega_n}G^{\rm KS}(k,\omega_n)\Sigma^{\rm SF}_{11}(k,k,\omega_n)G^{\rm KS}(k,\omega_n)
	= - \frac{1}{\beta^2}\sum_{k',\omega_n,\omega_m}
	3\Lambda_{kk'}(\omega_n-\omega_m)\frac{1}{({\rm i}\omega_n -\xi_{k})^2}\frac{1}{ {\rm i}\omega_m - \xi_{k'}},
	\label{eq:SF11sum1}
\end{equation}
%
\begin{equation}
	\begin{split}
	&\frac{1}{\beta} \sum_{\omega_n}G^{\rm KS}(k,\omega_n)\Sigma^{\rm SF}_{12}(k,k,\omega_n)G^{\rm KS}(k,-\omega_n)\\
	& \quad = -
	\frac{1}{4\beta^2}\sum_{k',\omega_n,\omega_m}\frac{\Delta^{\rm xc}_{k'}}{E_{k}E_{k'}}
	\left( \frac{1}{ {\rm i}\omega_n+E_k} - \frac{1}{ {\rm i}\omega_n-E_k} \right)
	\left( \frac{1}{ {\rm i}\omega_m+E_{k'}} - \frac{1}{ {\rm i}\omega_m-E_{k'}} \right)
	3\Lambda_{kk'}(\omega_n-\omega_m),
	\label{eq:SF12sum}
	\end{split}
\end{equation}
%
\begin{equation}
\begin{split}
	&\frac{1}{\beta} \sum_{\omega_n}G^{\rm KS}(k,\omega_n)\Sigma^{\rm SF}_{11}(k,k,\omega_n)F^{\rm KS}(k,\omega_n) \\
	&\quad =
	\frac{1}{\beta^2}\frac{\Delta^{\rm xc}_{k}}{2\xi_{k}}\sum_{k',\omega_n,\omega_m}
	\frac{1}{ {\rm i}\omega_n-\xi_k}\frac{1}{ {\rm i}\omega_m-\xi_{k'}}
	\left( \frac{1}{ {\rm i}\omega_n+\xi_k} - \frac{1}{ {\rm i}\omega_n-\xi_k} \right)
	3\Lambda_{kk'}(\omega_n-\omega_m).
	\label{eq:SF11sum2}
	\end{split}
\end{equation}
%
Inserting these expressions into (\ref{eq:Deltaxceq}) and taking the limit 
$\{\Delta_k\} \to 0$ properly, the SCDFT gap equation including the effect 
of spin fluctuations is derived straightforwardly.


\bibliographystyle{osajnl}
\bibliography{library_intro,library_method,library_result}

\end{document}

