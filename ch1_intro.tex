% まずはじめに \documentclass を指定する
\documentclass[uplatex]{jsbook}
%
% Packages
%
\usepackage[version=3]{mhchem}
\usepackage{geometry} % 余白の調整用
\usepackage{amsmath,amssymb}
\usepackage{booktabs} % 表組みのパッケージ
\usepackage{bm}
\usepackage[dvipdfmx]{graphicx}
\usepackage[subrefformat=parens]{subcaption}
\usepackage{ascmac}
\usepackage{braket} % ブラケット記法
\usepackage{cite} % [n-m]形式の引用を用いるため
%\usepackage{fancyhdr} % ヘッダ
\usepackage{feynmp} % Feynmanダイアグラムを書くため
\usepackage{color} % \debug 用に文字色をつける
\usepackage{setspace} % setstretchを使うため
\usepackage{newtxtext,newtxmath} % Times系フォントの使用

%\usepackage{abstract}
% 以下4つのパッケージはbibの著者名に出てきた欧文文字の文字化け対策として入れたが、
% 吉田は↓で上手くいく理由を良く理解していない。
% よく分からないパッケージは、それ無しで動くならコメントアウトしておくこと推奨。
\usepackage[uplatex,deluxe]{otf} %  \usepackage[prefernoncjk]{pxcjkcat} より先に読み込むべし(according to pxcjkcatのサイト)
\usepackage[prefernoncjk]{pxcjkcat}
\usepackage[T1]{fontenc} % T1エンコーディング, Bibliographyの著者用
%\usepackage[utf8]{inputenc}
%
% .bst ファイルは end.tex 内で指定することにしたから、次の行はコメントアウト
%\bibliographystyle{osajnlt}
%
% 行間と余白の調整
\setstretch{1.25}
\geometry{top=3truecm,bottom=3truecm,right=3truecm,left=3truecm}
%
% 見出しをセリフ・太字にする
\renewcommand{\headfont}{\bfseries}
%
% 脚注番号を記号に変える
\renewcommand{\thefootnote}{\fnsymbol{footnote}}
% 脚注記号を改ページでリセットする
\makeatletter
\@addtoreset{footnote}{page}
\makeatother
%
%「参考文献」を「References」にする
\renewcommand{\bibname}{References}
%「目次」を「Contents」にする
\renewcommand{\contentsname}{Contents}
%「第\CID{1624}章」を「Chapter x」にする
\renewcommand{\prechaptername}{Chapter }
\renewcommand{\postchaptername}{}

%表のキャプションを 表1.1ではなく Tabel1.1にする
\renewcommand{\tablename}{Table}
\renewcommand{\figurename}{Figure}


%% Appendix re-define
\renewcommand{\appendixname}{Appendix~}

%%目次にsabsectionを表示する
\setcounter{tocdepth}{2}

%
% 
\newcommand{\refeq}[1]{Eq.\,(\ref{#1})}
\renewcommand\vec\bm
\newcommand{\RR}{\vec{r}}
\newcommand{\PP}{\vec{p}}
\newcommand{\QQ}{\vec{q}}
\newcommand{\KK}{\vec{k}}
\newcommand{\TT}{\vec{t}}
\renewcommand{\SS}{\vec{s}}
\newcommand{\fermion}{\hat{\psi}}
\newcommand{\boson}{\hat{\phi}}
\newcommand{\BB}{\hat{b}}
\newcommand{\CC}{\hat{c}}
\newcommand{\momint}[1]{\frac{d^3{#1}}{(2\pi)^3}}
\newcommand{\sh}{\mathrm{Y}}
\newcommand{\schrodinger}{Schr\"{o}dinger }
\newcommand{\Tr}{\mathrm{Tr}}
\newcommand{\tr}{\mathrm{tr}}
\newcommand{\chrom}{\ce{Cr2O3}}
%
\newcommand{\debug}[1]{\textcolor{red}
{\textbf{[#1]}}}
\newcommand{\atom}[2][]{{}^{#1}\mathrm{#2}}
%
%
%
\begin{document}
% 数式の上下の余白を詰める
\setlength\abovedisplayskip{6.5pt}
\setlength\belowdisplayskip{6.5pt}
%
% Contents
%
\chapter{Introduction}



%%%%%%%%%%%%%%%%%%%%%   PURPOSE OF THIS STUDY %%%%%%%%%%%%%%%%
\section{History of theoretical studies on superconductors}
Superconductivity has been one of the most attracting topics in condensed matter physics since its
discovery in 1911\cite{Onnes1911}. Since then, physical properties of metals in low temperatures are
studied closely and it is found that various metals show the similar properties, 
zero resistivity, in low temperatures. In 1933\cite{Meissner1933}, it is found that
superconductors show the perfect diamagnetism which means that the external magnetic field is 
perfectly excluded from the superconductors. Theoretical studies to understand the microscopic 
mechanism of these macroscopic phenomena have been carried out.

Due to the physical and mathematical complexity, it took so long time until one successful
microscopic theory is proposed. Based on the insight that the electron-phonon interaction induces
the effective attraction between electrons\cite{Froehlich1950}, Bardeen, Cooper and Schrieffer(BCS)
constructed an 




\begin{equation} %%%%%%% electronic external potential
	\hat{V}^\mathrm{e}_\mathrm{ext} = \sum_{\sigma}\int d^3r\hat{\Psi}^\dag_{\sigma}({\bm r})
	                                                            v^\mathrm{e}_{\rm ext}(\bm{r})\hat{\Psi}_{\sigma}(\bm{r}), 
\end{equation} 

\include{end}
