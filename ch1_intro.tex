% まずはじめに \documentclass を指定する
\documentclass[uplatex]{jsbook}
%
% Packages
%
\usepackage[version=3]{mhchem}
\usepackage{geometry} % 余白の調整用
\usepackage{amsmath,amssymb}
\usepackage{booktabs} % 表組みのパッケージ
\usepackage{bm}
\usepackage[dvipdfmx]{graphicx}
\usepackage[subrefformat=parens]{subcaption}
\usepackage{ascmac}
\usepackage{braket} % ブラケット記法
\usepackage{cite} % [n-m]形式の引用を用いるため
%\usepackage{fancyhdr} % ヘッダ
\usepackage{feynmp} % Feynmanダイアグラムを書くため
\usepackage{color} % \debug 用に文字色をつける
\usepackage{setspace} % setstretchを使うため
\usepackage{newtxtext,newtxmath} % Times系フォントの使用

%\usepackage{abstract}
% 以下4つのパッケージはbibの著者名に出てきた欧文文字の文字化け対策として入れたが、
% 吉田は↓で上手くいく理由を良く理解していない。
% よく分からないパッケージは、それ無しで動くならコメントアウトしておくこと推奨。
\usepackage[uplatex,deluxe]{otf} %  \usepackage[prefernoncjk]{pxcjkcat} より先に読み込むべし(according to pxcjkcatのサイト)
\usepackage[prefernoncjk]{pxcjkcat}
\usepackage[T1]{fontenc} % T1エンコーディング, Bibliographyの著者用
%\usepackage[utf8]{inputenc}
%
% .bst ファイルは end.tex 内で指定することにしたから、次の行はコメントアウト
%\bibliographystyle{osajnlt}
%
% 行間と余白の調整
\setstretch{1.25}
\geometry{top=3truecm,bottom=3truecm,right=3truecm,left=3truecm}
%
% 見出しをセリフ・太字にする
\renewcommand{\headfont}{\bfseries}
%
% 脚注番号を記号に変える
\renewcommand{\thefootnote}{\fnsymbol{footnote}}
% 脚注記号を改ページでリセットする
\makeatletter
\@addtoreset{footnote}{page}
\makeatother
%
%「参考文献」を「References」にする
\renewcommand{\bibname}{References}
%「目次」を「Contents」にする
\renewcommand{\contentsname}{Contents}
%「第\CID{1624}章」を「Chapter x」にする
\renewcommand{\prechaptername}{Chapter }
\renewcommand{\postchaptername}{}

%表のキャプションを 表1.1ではなく Tabel1.1にする
\renewcommand{\tablename}{Table}
\renewcommand{\figurename}{Figure}


%% Appendix re-define
\renewcommand{\appendixname}{Appendix~}

%%目次にsabsectionを表示する
\setcounter{tocdepth}{2}

%
% 
\newcommand{\refeq}[1]{Eq.\,(\ref{#1})}
\renewcommand\vec\bm
\newcommand{\RR}{\vec{r}}
\newcommand{\PP}{\vec{p}}
\newcommand{\QQ}{\vec{q}}
\newcommand{\KK}{\vec{k}}
\newcommand{\TT}{\vec{t}}
\renewcommand{\SS}{\vec{s}}
\newcommand{\fermion}{\hat{\psi}}
\newcommand{\boson}{\hat{\phi}}
\newcommand{\BB}{\hat{b}}
\newcommand{\CC}{\hat{c}}
\newcommand{\momint}[1]{\frac{d^3{#1}}{(2\pi)^3}}
\newcommand{\sh}{\mathrm{Y}}
\newcommand{\schrodinger}{Schr\"{o}dinger }
\newcommand{\Tr}{\mathrm{Tr}}
\newcommand{\tr}{\mathrm{tr}}
\newcommand{\chrom}{\ce{Cr2O3}}
%
\newcommand{\debug}[1]{\textcolor{red}
{\textbf{[#1]}}}
\newcommand{\atom}[2][]{{}^{#1}\mathrm{#2}}
%
%
%

\begin{document}
% 数式の上下の余白を詰める
\setlength\abovedisplayskip{6.5pt}
\setlength\belowdisplayskip{6.5pt}
%
% Contents
%
\chapter{Introduction}
\label{intro}



%%%%%%%%%%%%%%%%%%%%%   PURPOSE OF THIS STUDY %%%%%%%%%%%%%%%%
\section{History of studies on conventional superconductors: Overview}
Since its discovery in 1911\cite{Onnes1911}, superconductivity has been one of the most fascinating
subject in the condensed matter physics. One of the characteristic properties of superconductors is
the zero resistivity and another is the Meissner effect\cite{Meissner1933}.
The latter means that bulk superconductors exclude the magnetic flux.
In 1950, Maxwell\cite{Maxwell1950} and Reynolds\cite{Reynolds1950} independently discovered the 
isotope effect for Hg,
%
\begin{equation}
	T_{c} \propto M^{-\alpha}, \alpha \approx 0.5,
	\label{eq:isotope}
\end{equation}
%
where $T_c$ is the superconducting transition temperature and $M$ indicates the mass of ion.
Although many results were presented experimentally soon after the discovery of superconductors,
it took a nearly half century until a successful microscopic theory for superconductors is presented.

In 1950, Fr\"{o}hlich\cite{Froehlich1950} demonstrated that the electron-phonon interaction induces the
effective attractive interaction between electrons. This idea is consistent with the isotope effect.
Based on this idea that the electronic condensation originates from the electron-phonon interaction, 
Bardeen, Cooper and Schrieffer(BCS)\cite{BCS1957} successfully constructed the microscopic 
theory by introducing the electronic many-body wave function which consists of electron pairs named 
Cooper pairs\cite{Cooper1956}. According to the BCS theory, $T_c$ is written as
%
\begin{equation}
	T_{c} \propto \omega\exp \left( -\frac{1}{N(0)V} \right),
	\label{eq:TcBCS}
\end{equation}
%
where $\omega, N(0), V$ are the typical phonon frequency, the density of states at the Fermi energy, 
and the characteristic attractive interaction between electrons, respectively.

In the BCS theory, the effective interaction between electrons is only assumed to be attractive and
the detail of the interaction is not studied.
In the late 1950s and 1960s, the detail of the phonon-mediated pairing interaction, namely the 
dynamical strucure of the electron-phonon interactinon, is studied by means of the Green's function 
formalism\cite{Migdal1958,Nambu1960,Eliashberg1960,Morel1962,Schrieffer1964,Scalapino1966}. 
This theory is called as Migdal-Eliashberg(ME) theory.
McMillan\cite{McMillan1968} solved the Eliashberg equations\cite{Parks1969}
and statistically derived an equation which describe $T_c$ with a simple analytic funcion. 
Later, the McMillan equation was improved by Allen and Dynes\cite{AllenDynes} and the resulting 
equation often called McMillan-Allen-Dynes formula is
%
\begin{equation}
	T_c = \frac{\omega_{\rm ln}}{1.2} \exp \left( -\frac{1.04(1+\lambda)}
	{\lambda-\mu^{\ast}(1+0.62\lambda)} \right),
	\label{eq:Allen}
\end{equation}
%
where $\omega_{\rm ln}$ indicates the logarithmic average of the typical phonon frequencies and
$\lambda$ is the dimensionless electron-phonon coupling. Finally, $\mu^{\ast}$ is the effective Coulomb
parater defined as
%
\begin{equation}
	\mu^{\ast} \equiv \frac{\mu}{1 + \mu \ln \left[ \frac{E_c}{\omega_{\rm ph}} \right]},
	\label{eq:mustar}
\end{equation}
%
where $\mu$ is the dimensionless Coulomb interaction parameter defined as the product of the density
of states at the Fermi energy and the averaged screened Coulomb interaction, $E_c$ is the
electronic band width and $\omega_{\rm ph}$ is the typical phonon frequency.
The reduction of the Coulomb parameter from $\mu$ to $\mu^{\ast}$ due to the difference of 
typical electronic and phononic energy range is called as retardation effect\cite{Bogo1958,Morel1962}.

%\section{Effect of spin fluctuations on the Eliashberg theory} %%%%% Title should be modified?
%The Eliashberg equations are believed to provide accurate results for $T_c$
%if reliable normal-state quantities, e.g. the electronic band structure, the electron-phonon coupling,
%phonon frequencies and the Coulomb interaction constant $\mu^{\ast}$, are given.
%However, it is found that, for some transition metals, $T_c$ obtained by solving the 
%Eliashberg equations is higher than experimental value by factor 2 or more.
%For example, the calculated $T_c$ with the above formula for both of Nb ($T_c = 9.2 \rm K$) 
%and V ($T_c = 5.3 \rm K$) is about 16K\cite{Papa1977}. 
%
%In 1979, Rietschel and Winter\cite{Rietschel1979} analyzed the Eliashberg
%equations including the effect of spin fluctuations(paramagnons). They employed the particle-hole 
%$t$-matrix\cite{Parks1969,Berk1966,Schrieffer1968} to consider the effect of spin fluctuations with 
%some parameters as follows
%
%\begin{equation}
%	\begin{split}
%		& \frac{m^{\ast}}{m} = 1 + \lambda_{\rm ph} + \lambda_{\rm spin}, \\
%		& \lambda_{\rm spin} = 2\int_{0}^{\infty} d\omega \frac{P(\omega)}{\omega}, \\
%		& P(\omega) = \frac{3N(0)}{2\pi} \int_{0}^{2k_{\rm F}} 
%		\frac{qdq}{2k_{\rm F}^2} {\rm Im} t(q, \omega),
%	\label{eq:eliash}
%\end{split}
%\end{equation}
%%\begin{equation}
%%	\Delta(\omega_{i}) = T_c\sum_{j}\frac{\pi}{|\tilde{\omega}_{j}}
%%	[\lambda^{-}(\omega_{i}-\omega_{j}) - \mu^{\ast}]\Delta(\omega_{j}),
%%	\label{}
%%\end{equation}
%%
%%\begin{figure}[h] %%%%%%% FIGURE 
%%\begin{minipage}[b]{0.5\linewidth}
%%	\centering
%%	\subcaption{}
%%	\includegraphics[keepaspectratio, scale=0.5]{../figure/intro/Tc_lambdaspin.eps}
%%	\label{fig:Tc_lambdaspin}
%%\end{minipage}
%%\begin{minipage}[b]{0.7\linewidth}
%%	\centering
%%	\subcaption{}
%%	\includegraphics[keepaspectratio, scale=0.5]{../figure/intro/Rietschel_table.eps}
%%	\label{fig:table}
%%\end{minipage}
%%\caption{(a) The paramagnon contribution $\lambda_{\rm spin}$ dependence of $T_c$ for Nb.
%%Open and closed circle correspond to values of the other parameter. (b) Experimental and theoretical
%%values for the Stoner factor $S$, $m^{\ast}/m$, phonon contribution $\lambda_{\rm ph}$  
%%and $\lambda_{\rm spin}$ for Nb and V.}
%%\end{figure}
%where $\lambda_{\rm spin}$ is the dimensionless electron-phonon coupling constant
%defined in the next chapter and $t(q,\omega)$ is the particle-hole $t$-matrix for which the
%random phase approximation\cite{Schrieffer1968} is applied. Two parameters which indicate the 
%intraatomic and the interatomic interaction is introduced in $t(q,\omega)$.
%It is found that if the effect of spin fluctuations is included, the resulting $T_c$ is reduced.
%They estimated the paramagnon contribution to the mass enhancement $m^{\ast}/m$ 
%by fitting the parameters in order to reproduce the resulting $T_c$ for Nb and V.
%Within this model, the resulting mass enhancement $m^{\ast}/m$ is in reasonable agreement 
%with experimental value for Nb. On the other hand, the calculated $m^{\ast}/m$ is little 
%overestimated for V.


\section{Density functional theory for superconductors}
\label{scdft}

In principle, one can obtain the reliable $T_c$ within the ME theory with reliable normal-state quantities.
However, there are some problems in the practice of the ME theory. First, there is an empirical parameter $\mu^{\ast}$
which represents the effect of the electronic Coulomb interaction suppressing the pairing. 
As long as there is an empirical parameter, 
the ME theory is not appropriate for quantitative prediction of $T_c$.
Another reason is that if we introduce $\mu^{\ast}$, it is implicitly assumed that electronic
Coulomb repulsion suppress the pairing of electrons. Within this restriction, we cannnot treat the
pairing originating from the repulsive interaction, e.g. plasmon mechanism and spin fluctuations.
To avoid the empirical treatment, a Green's function based formalism has been developed by Takada
\cite{Takada1978plasmon} to consider the plasmon-driven superconducting phase for homogeneous electron gas.

For the same purpose, an extension of the density functional theory for superconductors (SCDFT)
has been developed in 1988\cite{Oliveira1988}. Based on this theory, a numerical scheme to calculate
$T_c$ without introducing any empirical parameter has been implemented in 2005\cite{Luders2005} 
and applied to various kind of materials. We summarize some calculation results and experimental 
$T_c$ in Fig.\ref{fig:scdft_compare}. It has been shown that this scheme reproduces experimental $T_c$ 
of phonon-mediated conventional superconductors within an accuracy of a few Kelvin even if only electron-phonon 
interaction and static screened Coulomb interaction is considered within the level of the random
phase approximation.
It should be noted that there are some other applications of this scheme and its extension: 
Lithium under high pressure
\cite{Profeta2006Pressure, RA2013}, CaBeSi\cite{Bersier2009CaBeSi}, layered nitrides\cite{RA2012}, 
alkali-doped fullerides\cite{RA2013alkali}, recently discovered sulfur hydrides\cite{Flores2016,RA2015,Errea2015},
calculation of the order parameter and the condensation energy in real-space\cite{Linscheid2015},
and Fe-based superconductors\cite{Essenberger2016}.

In this scheme, the following gap equation analogous to that of the BCS theory is employed
%
\begin{equation}
	\Delta_{n \bm k} = -{\mathcal Z}_{n \bm k}\Delta_{n \bm k} - \frac{1}{2}
	\sum_{n' \bm k'} {\mathcal K}_{n \bm k n' \bm k'} \frac{\tanh[(\beta/2)E_{n' \bm k'}]}
	{E_{n' \bm k'}} \Delta_{n' \bm k'},
	\label{eq:gapeqintro}
\end{equation}
%
where $\beta$ is the inverse temperature, $\Delta$ is the gap function,
$n$ and $\bm k$ indicate the band index and the crystal momentum, 
$E_{n \bm k} $ is defined as $E_{n\bm k} = \sqrt{\xi_{n\bm k}^2 + \Delta_{n \bm k}^2}$ and 
$\xi_{n\bm k} = \varepsilon_{n\bm k} - \mu$ is the one-particle energy measured from the chemical
potential $\mu$. $\mathcal Z$ and $\mathcal K$ are the exchange-correlation kernels; the former
indicates the mass renormalization originating from the electron-phonon interaction, and the latter
includes the effect of the electron-phonon pairing interaction and the electronic Coulomb interaction.
This scheme also includes the phonon retardation effect through the exchange-correlation kernels.
This equation is much simpler than the Eliashberg equations and the exchange-correlation kernels 
can be calculated without introudcing any adjustable parameter such as $\mu^{\ast}$.
Therefore, we can carry out the {\it ab-initio} calculation of $T_c$ within this scheme.
The detail of the derivation of the SCDFT gap equation and the exchange-correlation kernels is 
summarized in the next chapter.

This numerical scheme has been extended to consider other contributions:
Particle-hole asymmetric electronic structure\cite{RA2013phasy}, dynamical screened Coulomb 
interaction\cite{RA2013}, and the effect of spin fluctuations\cite{Essenberger2014}.
These extensions also do not include the empirical parameters. 

\begin{figure} %%%%%%% FIGURE Coulombic free energy
	\centering
	\includegraphics[width=10truecm,clip]{../figure/intro/SCDFT_prev2.eps}
	\caption{Comparison of experimental and calculated $T_c$ 
		within SCDFT scheme\cite{Marques2005, Sanna2007, Floris2005}.}
	\label{fig:scdft_compare}
\end{figure}


\section{Effect of spin fluctuations in superconductors} %%%%% Title should be modified?
\subsection{Unconventional superconductors}
Recently, cuprate superconductors whose $T_c$ is surprisingly high has been discovered
\cite{Bednorz1986}. In 2006, Fe-based superconductor is discovered\cite{Kamihara2006} and the highest
$T_c$ of them is 56K\cite{wang2008}. On the other hand, in 1979 it has been found that $\rm CeCu_2Si_2$ 
shows the superconducting state below $T_c \approx 0.5\rm K$\cite{Steglich1979}. Since then the superconductibity in
heavy fermion compounds has been energetically explored and mainly Ce-based and U-based superconductors
has been discovered\cite{Pfleiderer2009}.
These novel superconducting phases cannnot be explained within the phonon-driven pairing mechanism and
the pairing mechanism is discussed\cite{Scalapino2012}.
A common feature among these novel superconductors is the proximity of the superconducting phase
and the magnetic ordered phase in phase diagrams.
According to this feature, it has been thought that the pairing mechanism in these unconventional
superconductors is related to spin fluctuations\cite{Scalapino2012}.


\subsection{Conventional superconductors}
The effect of spin fluctuations is also crucial in conventional phonon-driven superconductors.
If a metal has considerably enhanced magnetic susceptibility nearly enough to be a ferromagnet
due to the exchange interaction, the pairing of antiparallel spins is strongly suppressed because
parallel spin alignment is favored. This idea was proposed in case of Pd whose superconducting phase
is not observed\cite{Parks1969} and the ME theory has been extended to include the effect of spin 
fluctuations by means of the particle-hole $t$-matrix in the late 1960s\cite{Berk1966, Schrieffer1968}.
%The ME theory are believed to provide accurate results for $T_c$
%if reliable normal-state quantities, e.g. the electronic band structure, the electron-phonon coupling,
%phonon frequencies and the Coulomb interaction constant $\mu^{\ast}$, are given.
%However, in 1977, it is found that calculated $T_c$ within the ME theory is higher than experimental 
%value by factor 2 or more for some transition metals.
Later, it has been found that within the conventional ME theory using the rigid muffin-tin approximation, 
the calculated $T_c$ for both of Nb 
($T_c = 9.2 \rm K$) and V ($T_c = 5.3 \rm K$) is about 16K\cite{Papa1977}.
In 1979, Rietschel and Winter\cite{Rietschel1979} has proposed that the overestimation of $T_c$ for 
these materials comes from the neglected spin fluctuations effect and they employed the extended ME 
theory to consider the effect of spin fluctuations with some parameters. They calculated the $T_c$ 
for Nb and V with this extended theory and found that the effect of spin fluctuations somehow suppresses $T_c$. 

\section{Motivation and outline}
%As mentioned in the previous section, the effect of spin fluctuations on $T_c$ of conventional 
%superconductors is studied within the ME theory with some adjustable parameters.
%On the other hand, the extended version of SCDFT including the effect of spin fluctuations was 
%recently developed\cite{Essenberger2014} and applied to Fe-based superconductors\cite{Essenberger2016}.
%It should be noted that in application to Fe-based superconductors, it is needed to introduce a
%scaling parameter in order to avoid the singulality corresponding to the phase transition to the 
%magnetic ordered phase.
%However, non-empirical studies on the effect of spin fluctuations in conventional superconductors
%and quantitative estimation of the reduction of $T_c$ is lacking.

As mentioned in the section \ref{scdft}, SCDFT has enabled us to calculate the $T_c$ of various kinds of conventional
superconductors with an accuracy of a few Kelvin. 
For this reason, first-principles calculations of $T_c$ with such accuracy become more significant.
%However, the problem that $T_c$ for some transition metals is 
%considerably overestimated within the ME theory is still remained. 
However, the effect of spin fluctuations on $T_c$ of conventional superconductors has scarcely been 
studied from first-principles.
In order to confirm that SCDFT
can describe $T_c$ for such materials in sufficient accuracy like other conventional superconductors,
first-principles calculations of $T_c$ including the effect of spin fluctuations within SCDFT is 
worth carrying out.

Therefore, we applied the recently developed scheme\cite{Essenberger2014} on $T_c$ of Nb and V 
whose $T_c$ is overestimated within the ME theory, and Al whose $T_c$ is reproduced within 
SCDFT (Fig.\ref{fig:scdft_compare}) in sufficient accuracy. We also studied the difference of the 
effect of spin fluctuations among these materials.

In Chapter \ref{theory} we briefly review the theorical foundation of the numerical scheme 
developed by L\"{u}ders and his coworkers\cite{Luders2005} and its extension 
including the effect of spin fluctuations\cite{Essenberger2014}.
In Chapter \ref{application} we apply the extended scheme to elemental metals Nb, V and Al and show 
the calculation results and compare the results between materials.
In Chapter \ref{conclusion} we conclude this thesis with perspectives.


%\begin{equation}
%	\begin{split}
%		& \frac{m^{\ast}}{m} = 1 + \lambda_{\rm ph} + \lambda_{\rm spin}, \\
%		& \lambda_{\rm spin} = 2\int_{0}^{\infty} d\omega \frac{P(\omega)}{\omega}, \\
%		& P(\omega) = \frac{3N(0)}{2\pi} \int_{0}^{2k_{\rm F}} 
%		\frac{qdq}{2k_{\rm F}^2} {\rm Im} t(q, \omega),
%	\label{eq:eliash}
%\end{split}
%\end{equation}
%\begin{equation}
%	\Delta(\omega_{i}) = T_c\sum_{j}\frac{\pi}{|\tilde{\omega}_{j}}
%	[\lambda^{-}(\omega_{i}-\omega_{j}) - \mu^{\ast}]\Delta(\omega_{j}),
%	\label{}
%\end{equation}
%
%\begin{figure}[h] %%%%%%% FIGURE 
%\begin{minipage}[b]{0.5\linewidth}
%	\centering
%	\subcaption{}
%	\includegraphics[keepaspectratio, scale=0.5]{../figure/intro/Tc_lambdaspin.eps}
%	\label{fig:Tc_lambdaspin}
%\end{minipage}
%\begin{minipage}[b]{0.7\linewidth}
%	\centering
%	\subcaption{}
%	\includegraphics[keepaspectratio, scale=0.5]{../figure/intro/Rietschel_table.eps}
%	\label{fig:table}
%\end{minipage}
%\caption{(a) The paramagnon contribution $\lambda_{\rm spin}$ dependence of $T_c$ for Nb.
%Open and closed circle correspond to values of the other parameter. (b) Experimental and theoretical
%values for the Stoner factor $S$, $m^{\ast}/m$, phonon contribution $\lambda_{\rm ph}$  
%and $\lambda_{\rm spin}$ for Nb and V.}
%\end{figure}
%where $\lambda_{\rm spin}$ is the dimensionless electron-phonon coupling constant
%defined in the next chapter and $t(q,\omega)$ is the particle-hole $t$-matrix for which the
%random phase approximation\cite{Schrieffer1968} is applied. Two parameters which indicate the 
%intraatomic and the interatomic interaction is introduced in $t(q,\omega)$.
%It is found that if the effect of spin fluctuations is included, the resulting $T_c$ is reduced.
%They estimated the paramagnon contribution to the mass enhancement $m^{\ast}/m$ 
%by fitting the parameters in order to reproduce the resulting $T_c$ for Nb and V.
%Within this model, the resulting mass enhancement $m^{\ast}/m$ is in reasonable agreement 
%with experimental value for Nb. On the other hand, the calculated $m^{\ast}/m$ is little 
%overestimated for V.


\bibliographystyle{osajnl}
\bibliography{library_intro,library_method,library_result}

\end{document}
