% まずはじめに \documentclass を指定する
\documentclass[uplatex]{jsbook}
%
% Packages
%
\usepackage[version=3]{mhchem}
\usepackage{geometry} % 余白の調整用
\usepackage{amsmath,amssymb}
\usepackage{booktabs} % 表組みのパッケージ
\usepackage{bm}
\usepackage[dvipdfmx]{graphicx}
\usepackage[subrefformat=parens]{subcaption}
\usepackage{ascmac}
\usepackage{braket} % ブラケット記法
\usepackage{cite} % [n-m]形式の引用を用いるため
%\usepackage{fancyhdr} % ヘッダ
\usepackage{feynmp} % Feynmanダイアグラムを書くため
\usepackage{color} % \debug 用に文字色をつける
\usepackage{setspace} % setstretchを使うため
\usepackage{newtxtext,newtxmath} % Times系フォントの使用

%\usepackage{abstract}
% 以下4つのパッケージはbibの著者名に出てきた欧文文字の文字化け対策として入れたが、
% 吉田は↓で上手くいく理由を良く理解していない。
% よく分からないパッケージは、それ無しで動くならコメントアウトしておくこと推奨。
\usepackage[uplatex,deluxe]{otf} %  \usepackage[prefernoncjk]{pxcjkcat} より先に読み込むべし(according to pxcjkcatのサイト)
\usepackage[prefernoncjk]{pxcjkcat}
\usepackage[T1]{fontenc} % T1エンコーディング, Bibliographyの著者用
%\usepackage[utf8]{inputenc}
%
% .bst ファイルは end.tex 内で指定することにしたから、次の行はコメントアウト
%\bibliographystyle{osajnlt}
%
% 行間と余白の調整
\setstretch{1.25}
\geometry{top=3truecm,bottom=3truecm,right=3truecm,left=3truecm}
%
% 見出しをセリフ・太字にする
\renewcommand{\headfont}{\bfseries}
%
% 脚注番号を記号に変える
\renewcommand{\thefootnote}{\fnsymbol{footnote}}
% 脚注記号を改ページでリセットする
\makeatletter
\@addtoreset{footnote}{page}
\makeatother
%
%「参考文献」を「References」にする
\renewcommand{\bibname}{References}
%「目次」を「Contents」にする
\renewcommand{\contentsname}{Contents}
%「第\CID{1624}章」を「Chapter x」にする
\renewcommand{\prechaptername}{Chapter }
\renewcommand{\postchaptername}{}

%表のキャプションを 表1.1ではなく Tabel1.1にする
\renewcommand{\tablename}{Table}
\renewcommand{\figurename}{Figure}


%% Appendix re-define
\renewcommand{\appendixname}{Appendix~}

%%目次にsabsectionを表示する
\setcounter{tocdepth}{2}

%
% 
\newcommand{\refeq}[1]{Eq.\,(\ref{#1})}
\renewcommand\vec\bm
\newcommand{\RR}{\vec{r}}
\newcommand{\PP}{\vec{p}}
\newcommand{\QQ}{\vec{q}}
\newcommand{\KK}{\vec{k}}
\newcommand{\TT}{\vec{t}}
\renewcommand{\SS}{\vec{s}}
\newcommand{\fermion}{\hat{\psi}}
\newcommand{\boson}{\hat{\phi}}
\newcommand{\BB}{\hat{b}}
\newcommand{\CC}{\hat{c}}
\newcommand{\momint}[1]{\frac{d^3{#1}}{(2\pi)^3}}
\newcommand{\sh}{\mathrm{Y}}
\newcommand{\schrodinger}{Schr\"{o}dinger }
\newcommand{\Tr}{\mathrm{Tr}}
\newcommand{\tr}{\mathrm{tr}}
\newcommand{\chrom}{\ce{Cr2O3}}
%
\newcommand{\debug}[1]{\textcolor{red}
{\textbf{[#1]}}}
\newcommand{\atom}[2][]{{}^{#1}\mathrm{#2}}
%
%
%

\begin{document}
% 数式の上下の余白を詰める
\setlength\abovedisplayskip{6.5pt}
\setlength\belowdisplayskip{6.5pt}
%
% Contents
%
\chapter{Introduction}



%%%%%%%%%%%%%%%%%%%%%   PURPOSE OF THIS STUDY %%%%%%%%%%%%%%%%
\section{History of studies on conventional superconductors: Overview}
Since its discovery in 1911\cite{Onnes1911}, superconductivity has been one of the most fascinating
subject in the condensed matter physics. One of the characteristic properties of superconductors is
the zero resistivity and another is the Meissner effect\cite{Meissner1933}.
The latter means that bulk superconductors exclude the magnetic flux.
In 1950, Maxwell\cite{Maxwell1950} and Reynolds\cite{Reynolds1950} independently discovered the 
isotope effect for Hg,
%
\begin{equation}
	T_{c} \propto M^{-\alpha}, \alpha \approx 0.5,
	\label{eq:isotope}
\end{equation}
%
where $T_c$ is the superconducting transition temperature and $M$ indicates the mass of ion.
Although many results were presented experimentally soon after the discovery of superconductors,
it took a nearly half century until a successful microscopic theory for superconductors is presented.

In 1950, Fr\"{o}hlich\cite{Froehlich1950} demonstrated that the electron-phonon interaction induces the
effective attractive interaction between electrons. This idea is consistent to the isotope effect.
Based on this idea that the electronic condensation originates from the electron-phonon interaction, 
Bardeen, Cooper and Schrieffer(BCS)\cite{BCS1957} successfully constructed the microscopic 
theory by introducing the electronic many-body wave function which consist of electron pairs named as
Cooper pairs\cite{Cooper1956}. According to the BCS theory, $T_c$ is written as
%
\begin{equation}
	T_{c} \propto \omega\exp \left( -\frac{1}{N(0)V} \right),
	\label{eq:TcBCS}
\end{equation}
%
where $\omega, N(0), V$ is the typical phonon frequency, the density of states at the Fermi energy, 
and the characteristic attractive interaction between electrons respectively.

In the BCS theory, the effective interaction between electrons is only assumed to be attractive and
the detail of the interaction is not studied.
In the late 1950s and 1960s, the detail of the phonon-mediated pairing interaction, namely the 
dynamical strucure of the electron-phonon interactinon, is studied by means of the Green's function 
formalism\cite{Migdal1958,Nambu1960,Eliashberg1960,Morel1962,Schrieffer1964,Scalapino1966}. 
This theory is called as Migdal-Eliashberg(ME) theory.
McMillan\cite{McMillan1968} solved the Eliashberg equations\cite{Parks1969}
and statistically derived an equation which describe $T_c$ with a simple analytic funcion. 
Later, the McMillan's equation was improved by Allen and Dynes\cite{AllenDynes} and the resulting 
equation often called McMillan-Allen-Dynes formula is
%
\begin{equation}
	T_c = \frac{\omega_{\rm ln}}{1.2} \exp \left( -\frac{1.04(1+\lambda)}
	{\lambda-\mu^{\ast}(1+0.62\lambda)} \right),
	\label{eq:Allen}
\end{equation}
%
where $\omega_{\rm ln}$ indicates the logarithmic average of the typical phonon frequencies and
$\lambda$ is the dimensionless electron-phonon coupling. Finally, $\mu^{\ast}$ is the effective Coulomb
parater defined as
%
\begin{equation}
	\mu^{\ast} \equiv \frac{\mu}{1 + \mu \ln \left[ \frac{E_c}{\omega_{\rm ph}} \right]},
	\label{eq:mustar}
\end{equation}
%
where $\mu$ is the dimensionless Coulomb interaction parameter defined as the product of the density
of states at the Fermi energy and the averaged screened Coulomb interaction, $E_c$ is the
electronic band width and $\omega_{\rm ph}$ is the typical phonon frequency.
The reduction of the Coulomb parameter from $\mu$ to $\mu^{\ast}$ due to the difference of 
typical electronic and phononic energy range is called as retardian effect\cite{Bogo1958,Morel1962}.

%\section{Effect of spin fluctuations on the Eliashberg theory} %%%%% Title should be modified?
%The Eliashberg equations are believed to provide accurate results for $T_c$
%if reliable normal-state quantities, e.g. the electronic band structure, the electron-phonon coupling,
%phonon frequencies and the Coulomb interaction constant $\mu^{\ast}$, are given.
%However, it is found that, for some transition metals, $T_c$ obtained by solving the 
%Eliashberg equations is higher than experimental value by factor 2 or more.
%For example, the calculated $T_c$ with the above formula for both of Nb ($T_c = 9.2 \rm K$) 
%and V ($T_c = 5.3 \rm K$) is about 16K\cite{Papa1977}. 
%
%In 1979, Rietschel and Winter\cite{Rietschel1979} analyzed the Eliashberg
%equations including the effect of spin fluctuations(paramagnons). They employed the particle-hole 
%$t$-matrix\cite{Parks1969,Berk1966,Schrieffer1968} to consider the effect of spin fluctuations with 
%some parameters as follows
%
%\begin{equation}
%	\begin{split}
%		& \frac{m^{\ast}}{m} = 1 + \lambda_{\rm ph} + \lambda_{\rm spin}, \\
%		& \lambda_{\rm spin} = 2\int_{0}^{\infty} d\omega \frac{P(\omega)}{\omega}, \\
%		& P(\omega) = \frac{3N(0)}{2\pi} \int_{0}^{2k_{\rm F}} 
%		\frac{qdq}{2k_{\rm F}^2} {\rm Im} t(q, \omega),
%	\label{eq:eliash}
%\end{split}
%\end{equation}
%%\begin{equation}
%%	\Delta(\omega_{i}) = T_c\sum_{j}\frac{\pi}{|\tilde{\omega}_{j}}
%%	[\lambda^{-}(\omega_{i}-\omega_{j}) - \mu^{\ast}]\Delta(\omega_{j}),
%%	\label{}
%%\end{equation}
%%
%%\begin{figure}[h] %%%%%%% FIGURE 
%%\begin{minipage}[b]{0.5\linewidth}
%%	\centering
%%	\subcaption{}
%%	\includegraphics[keepaspectratio, scale=0.5]{../figure/intro/Tc_lambdaspin.eps}
%%	\label{fig:Tc_lambdaspin}
%%\end{minipage}
%%\begin{minipage}[b]{0.7\linewidth}
%%	\centering
%%	\subcaption{}
%%	\includegraphics[keepaspectratio, scale=0.5]{../figure/intro/Rietschel_table.eps}
%%	\label{fig:table}
%%\end{minipage}
%%\caption{(a) The paramagnon contribution $\lambda_{\rm spin}$ dependence of $T_c$ for Nb.
%%Open and closed circle correspond to values of the other parameter. (b) Experimental and theoretical
%%values for the Stoner factor $S$, $m^{\ast}/m$, phonon contribution $\lambda_{\rm ph}$  
%%and $\lambda_{\rm spin}$ for Nb and V.}
%%\end{figure}
%where $\lambda_{\rm spin}$ is the dimensionless electron-phonon coupling constant
%defined in the next chapter and $t(q,\omega)$ is the particle-hole $t$-matrix for which the
%random phase approximation\cite{Schrieffer1968} is applied. Two parameters which indicate the 
%intraatomic and the interatomic interaction is introduced in $t(q,\omega)$.
%It is found that if the effect of spin fluctuations is included, the resulting $T_c$ is reduced.
%They estimated the paramagnon contribution to the mass enhancement $m^{\ast}/m$ 
%by fitting the parameters in order to reproduce the resulting $T_c$ for Nb and V.
%Within this model, the resulting mass enhancement $m^{\ast}/m$ is in reasonable agreement 
%with experimental value for Nb. On the other hand, the calculated $m^{\ast}/m$ is little 
%overestimated for V.


\section{Density functional theory for superconductors}

One can obtain the reliable $T_c$ within the ME theory with reliable normal-state quantities.
However, there are some problems in the ME theory. First, there is an empirical parameter $\mu^{\ast}$
which means the electronic Coulomb interaction. As long as there is an empirical parameter, 
this theory is not appropriate to predict $T_c$ quantitatively.
Another reason is that if we introduce $\mu^{\ast}$, it is implicitly assumed that electronic
Coulomb repulsion suppress the pairing of electrons. Within this restriction, we can not treat the
pairing originating from the repulsive interaction, e.g. plasmon mechanism and spin fluctuations.
To avoid introducing the empirical parametrer, the Green's function formalism is developed by Takada
\cite{Takada1978plasmon} to consider the plasmon-driven superconducting phase for homogeneous electron gas.

On the same purpose, an extension of the density functional theory for superconductors (SCDFT)
is developed in 1988\cite{Oliveira1988}. Based on this idea, a numerical scheme to calculate
$T_c$ without introducing any empirical parameter is implemented in 2005\cite{Luders2005} 
and applied to various kind of materials. We summarized some calculation results and experimental 
$T_c$ in Fig.\ref{fig:scdft_compare}. It has been shown that this scheme reproduce experimental $T_c$ 
of phonon-mediated conventional superconductors within about 20\% even if only electron-phonon 
interaction and static screened Coulomb interaction is considered.
This scheme is energetically extended to consider other contributions:
Particle-hole asymmetric electronic structure\cite{RA2013phasy}, dynamical screened Coulomb 
interaction\cite{RA2013}, and the effect of spin fluctuations\cite{Essenberger2014}.
It should be noted that there are some other applications of this scheme and its extension: 
Lithium under high pressure
\cite{Profeta2006Pressure, RA2013}, CaBeSi\cite{Bersier2009CaBeSi}, layered nitrides\cite{RA2012}, 
alkali-doped fullerides\cite{RA2013alkali}, and recently discovered sulfur hydrides\cite{RA2015}.

\begin{figure} %%%%%%% FIGURE Coulombic free energy
	\centering
	\includegraphics[width=10truecm,clip]{../figure/intro/SCDFT_prev2.eps}
	\caption{Comparison of experimental and calculated $T_c$ 
		within SCDFT scheme\cite{Marques2005, Sanna2007, Floris2005}.}
	\label{fig:scdft_compare}
\end{figure}


\section{Effect of spin fluctuations in superconductors} %%%%% Title should be modified?
Recently, surprisingly high-$T_c$ cuprate superconductor whose $T_c$ is 153K is discovered
\cite{Bednorz1986}. In 2008, Fe-based superconductor is discovered\cite{Kamihara2008} and the highest
$T_c$ of them is 56K\cite{wang2008}. These high $T_c$ can not be explained with the phonon-driven
pairing mechanism. The origin of the pairing in these materials is energetically discussed
\cite{Scalapino2012} and one of the possible mechanisms is spin fluctuations.

However, the effect of spin fluctuations is also crucial in conventional phonon-driven superconductors.
The ME theory are believed to provide accurate results for $T_c$
if reliable normal-state quantities, e.g. the electronic band structure, the electron-phonon coupling,
phonon frequencies and the Coulomb interaction constant $\mu^{\ast}$, are given.
However, in 1977, it is found that calculated $T_c$ within the ME theory is higher than experimental 
value by factor 2 or more for some transition metals.
In particular, the calculated $T_c$ for both of Nb ($T_c = 9.2 \rm K$) and V ($T_c = 5.3 \rm K$) 
is about 16K\cite{Papa1977}. In 1979, Rietschel and Winter\cite{Rietschel1979} analyzed the Eliashberg
equations including the effect of spin fluctuations(paramagnons). They employed the particle-hole 
$t$-matrix\cite{Parks1969,Berk1966,Schrieffer1968} to consider the effect of spin fluctuations with 
some parameters. They calculated the $T_c$ for Nb and V with this extended theory and found that 
the effect of spin fluctuations suppresses $T_c$. 

%\begin{equation}
%	\begin{split}
%		& \frac{m^{\ast}}{m} = 1 + \lambda_{\rm ph} + \lambda_{\rm spin}, \\
%		& \lambda_{\rm spin} = 2\int_{0}^{\infty} d\omega \frac{P(\omega)}{\omega}, \\
%		& P(\omega) = \frac{3N(0)}{2\pi} \int_{0}^{2k_{\rm F}} 
%		\frac{qdq}{2k_{\rm F}^2} {\rm Im} t(q, \omega),
%	\label{eq:eliash}
%\end{split}
%\end{equation}
%\begin{equation}
%	\Delta(\omega_{i}) = T_c\sum_{j}\frac{\pi}{|\tilde{\omega}_{j}}
%	[\lambda^{-}(\omega_{i}-\omega_{j}) - \mu^{\ast}]\Delta(\omega_{j}),
%	\label{}
%\end{equation}
%
%\begin{figure}[h] %%%%%%% FIGURE 
%\begin{minipage}[b]{0.5\linewidth}
%	\centering
%	\subcaption{}
%	\includegraphics[keepaspectratio, scale=0.5]{../figure/intro/Tc_lambdaspin.eps}
%	\label{fig:Tc_lambdaspin}
%\end{minipage}
%\begin{minipage}[b]{0.7\linewidth}
%	\centering
%	\subcaption{}
%	\includegraphics[keepaspectratio, scale=0.5]{../figure/intro/Rietschel_table.eps}
%	\label{fig:table}
%\end{minipage}
%\caption{(a) The paramagnon contribution $\lambda_{\rm spin}$ dependence of $T_c$ for Nb.
%Open and closed circle correspond to values of the other parameter. (b) Experimental and theoretical
%values for the Stoner factor $S$, $m^{\ast}/m$, phonon contribution $\lambda_{\rm ph}$  
%and $\lambda_{\rm spin}$ for Nb and V.}
%\end{figure}
%where $\lambda_{\rm spin}$ is the dimensionless electron-phonon coupling constant
%defined in the next chapter and $t(q,\omega)$ is the particle-hole $t$-matrix for which the
%random phase approximation\cite{Schrieffer1968} is applied. Two parameters which indicate the 
%intraatomic and the interatomic interaction is introduced in $t(q,\omega)$.
%It is found that if the effect of spin fluctuations is included, the resulting $T_c$ is reduced.
%They estimated the paramagnon contribution to the mass enhancement $m^{\ast}/m$ 
%by fitting the parameters in order to reproduce the resulting $T_c$ for Nb and V.
%Within this model, the resulting mass enhancement $m^{\ast}/m$ is in reasonable agreement 
%with experimental value for Nb. On the other hand, the calculated $m^{\ast}/m$ is little 
%overestimated for V.


\bibliographystyle{osajnl}
\bibliography{library_intro,library_method,library_result}

\end{document}
